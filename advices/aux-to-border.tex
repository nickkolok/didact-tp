\begin{notbadadvice}[Вспомогательные конструкции --- на края программы]
	\label{NA_aux-to-border}
	Следует чётко отличать вспомогательные части программы (чаще всего --- операторы) от основных.
	Наиболее простой, хотя и не всегда точный, критерий отделения таков:
	вспомогательные конструкции не отражаются в блок-схеме, подобно тому,
	как вампиры не отражаются в зеркале.
	Сразу же, однако, надо сделать оговорку: объявление переменной --- не вспомогательная конструкция!
	К вспомогательным можно отнести, например, вызов \textbf{setlocale}.

	Вспомогательные конструкции следует сдвигать как можно ближе к краю программы.
	Это простое правило позволит существенно облегчить восприятие кода:
	действительно нужные, содержательные действия не будет смешиваться с ничего не значащими строчками.

	Неправильно:
	\codesnippet{../advices-snippets/aux-to-border-1-wrong}

	Здесь строчка \textbf{setlocale(LC\_ALL,0);} делает чисто техническую вещь ---
	устанавливает адекватную кодировку (если Вам не повезло и Вы работаете под windows).
	Ей не место посреди алгоритма, он лишь смущает человека, читающего код.
	В том числе и того, кто этот код пишет.

	Лучше всего отбросить её к краю программы --- в нашем случае, в начало функции \textbf{main}.
	И отделить пустой строчкой.

	Правильно:
	\codesnippet{../advices-snippets/aux-to-border-1-right}

\end{notbadadvice}
