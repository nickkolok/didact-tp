\begin{notbadadvice}[Объявляйте счётчики внутри цикла, когда это возможно]
	\label{NA_hide-your-counters}
	
	Рассмотрим два фрагмента программного кода, работающих одинаково и
	отличающихся только объявлением переменной-счётчика:

	Вне цикла:
	\codesnippet{../advices-snippets/hide-your-counters-1-outer}

	Внутри цикла:
	\codesnippet{../advices-snippets/hide-your-counters-1-inner}

	С одной стороны, вариант с объявлением переменной вне цикла кажется более экономным по памяти.
	Более того, программисту, переходящему с Паскаля, привычнее,
	когда все переменные объявлены в начале программы.
	С другой стороны, современные компиляторы умеют оптимизировать код,
	так что по памяти вариант с объявлением два раза едва ли проиграет.
	
	Решающий же аргумент в пользу объявления переменной в цикле ---
	то, что такую переменную не видно вне цикла.
	Сравним:
	
	Вне цикла:
	\codesnippet{../advices-snippets/hide-your-counters-2-outer}

	Внутри цикла:
	\codesnippet{../advices-snippets/hide-your-counters-2-inner}

	Польза от такой особенности неочевидна, но она есть:
	к переменной, по которой идёт итерирование цикла, невозможно обратиться случайно.
	Программист, обращаясь со счётчиком в цикле, чётко отдаёт себе отчёт в том,
	нужен ли ему будет этот счётчик вне цикла.
	Иногда бывает полезно некоторые переменные делать внутренними, а некоторые --- внешними.
	
	Рассмотрим, например, следующую программу, которая запрашивает у пользователя массив,
	затем удаляет из него все отрицательные элементы и выводит на экран:
	\codesnippet{../advices-snippets/hide-your-counters-3-mixed}	
	
	Счётчики \textbf{i} в обоих циклах являются внутренними,
	т.е. их значения не используются вне циклов, в которых эти счётчики объявлены.
	Со счётчиком \textbf{j} ситуация иная:
	его значение в конце цикла в точности равно количеству элементов в массиве после удаления,
	которое, очевидно, потребуется вне цикла.
	Действительно, именно от \textbf{0} до \textbf{j} изменяется счётчик \textbf{k}
	в последнем цикле. используемом для вывода.
	Кстати, счётчик в последнем цикле можно снова было назвать \textbf{i}.
	Всё равно этот счётчик не виден вне цикла и не конфликтует с другими счётчиками.
	
	Подробнее об этом эффекте можно узнать, поискав информацию об областях видимости.
	
	Следование этому совету требует определённых усилий, но эти усилия окупаются
	улучшением читаемости программы.

\end{notbadadvice}
