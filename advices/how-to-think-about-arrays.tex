\begin{notbadadvice}[Как думать о задачах с массивами]
	\label{NA_how-to-think-about-arrays}

	Как известно, ванильные первокурсницы сидят на парах, пьют малиновый чай и думают о лабораторных работах.

	Если в задаче нужно преобразовать массив, то полезно начинать размышления следующим образом:
	\begin{enumerate}
		\item
			Найдите листочек бумаги в клеточку.
		\item
			Нарисуйте исходный массив и массив, который должен получиться.
			Старайтесь умещать каждый элемент массива в отдельную клеточку.
		\item
			Если возможно, используйте в качестве исходного массива массив чисел от 0 до 5 или 6.
		\item
			Не гонитесь за требуемой длиной.
			Увеличить размер всегда успеете.
			%TODO: ссылка на НС про вынос размера в константу.
		\item
			Если в задаче требуется изменить исходный массив, а не сформировать новый,
			но никаких мыслей в голову не приходит, думайте так, как будто это два разных массива.
			В крайнем случае --- приравнять исходный полученному всегда успеете.
		\item
			Подумайте о том, как связана между собой длина исходного и полученного массива.
			Можете ли Вы сразу, не приступая к обработке, назвать длину получаемого массива?
		\item
			Если работа с массивом, элементы которого равны индексам, вызывает трудности,
			прибавьте к значениям какое-нибудь число.
			Например, подумайте о целочисленном массиве длины 6 с элементами от 10 до 15.
			Если же нужно иметь дело с массивом нецелых чисел, и прибавить одинаковые дробные части нельзя,
			но прибавляйте разные.
			Например, может получиться такой массив: 10.1 11.3 12.5 13.0 14.76 15.99.
			Глядя на каждый его элемент, Вы в любой момент сможете легко назвать его индекс.
		\item
			Помните, что задачу, в которой требуется удалить элементы из массива, всегда можно свести к задаче,
			в которой нужно сформировать новый массив из элементов исходного.
		\item
			Посмотрите на выписанные массивы.
			Пусть $a$ --- исходный, $b$ --- полученный.
			Можете ли Вы в явном виде выразить элемент $b_i$, например,
			$$
				b_i = \left\{\begin{array}{ll}
					a_i,    & \mbox{ если } i<3 \\
					a_{i+1} & \mbox{иначе}
				\end{array}\right.
			$$
			(это правило соответствует удалению третьего элемента).
	\end{enumerate}
\end{notbadadvice}
