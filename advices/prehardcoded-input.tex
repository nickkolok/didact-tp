\begin{notbadadvice}[Фиксируйте отладочные данные]
	\label{NA_prehardcoded-input}

	Фиксация отладочных данных --- это полезнейший приём,
	который сэкономит уйму времени на отладке программисту, этот приём освоившему.
	Экономия эта происходит не за счёт того, чтобы меньше думать,
	а за счёт того, чтобы меньше повторять технические действия.
	Иначе говоря, <<программист работает не руками, программист работает головой>>.

	Часто при отладке, т.е. при упорных попытках заставить программу работать правильно
	(а иногда - и просто хоть как-то работать),
	приходится много раз вводить одни и те же данные.

	Например, требуется написать программу, которая обрабатывает массив.
	Программист пишет первый вариант программы, запускает его, вводит массив,
	возможно, вводит какие-то дополнительные данные
	(количество обрабатываемых элементов, число для сравнения и т.д.)...
	и обнаруживает, что результат неправильный.
	Подержавшись минут пять за голову и горестно повздыхав,
	программист исправляет, скажем, строгое равенство на нестрогое,
	снова запускает программу, вводит массив
	(который надо ещё и помнить, чтобы ввести точно такой же, как в прошлый раз)...
	И запросто может снова получить ошибку.

	Однако, этого можно избежать, если заменить реальный ввод массива с клавиатуры на фиктивный.
	Эта процедура называется фиксацией отладочных данных.

	Примеры даны в листингах №№
	\ref{prehardcoded-input},
	\ref{prehardcoded-input-variants},
	\ref{prehardcoded-input-array1d-void},
	\ref{prehardcoded-input-array2d}.

	%\codesnippet{../advices-snippets/hide-your-counters-1-outer}
	%\codesnippet{../advices-snippets/hide-your-counters-1-inner}

\end{notbadadvice}
