\chapter*{О чём эта книга?}

Книга, которую Вы держите в руках ---
или, что вероятнее, видите на мониторе ---
в большей своей части есть не что иное, как средство автоматизации.

Её написание начиналось с составления лабораторных работ,
которые по доброй линуксоидной традиции были аккуратно сложены в git-репозиторий.
По мере выполнения лабораторных работ студентами выяснилось,
что каждый раз повторять требования к работам банально неудобно,
и в качестве автоматизирующего решения был написан раздел <<Требования к лабораторным работам>>.

При проверке лабораторных работ было замечено, что определённые ошибки совершаются студентами достаточно часто.
Эти ошибки авторы назвали типовыми и вынесли в соответствующий раздел.
Нумерация ошибок является хронологической, т.е. номер присваивался по мере обнаружения.
Стоит заметить, что написание этого раздела сэкономило авторам массу времени:
вместо того, чтобы много раз объяснять одно и то же разным студентам,
достаточно было один раз написать, при отклонении лабораторной работы
--- ссылаться на номер типовой ошибки, а при непонимании студентом сути ---
дополнять описание (и объяснять <<вручную>> в совсем запущенных случаях).
Возьмём на себя смелость предположить,
что именно раздел с типовыми ошибками имеет наибольшую ценность для преподавателей программирования на языке С++,
так как позволяет сократить время, затрачиваемое на объяснение материала.

На разделе <<Листинги программ>> особо останавливаться не будем ---
он почти полностью кумулятивен;
<<Правила оформления программного кода>> содержат мотивировочную часть внутри раздела,
и здесь мы просто отсылаем читателя к ней.

Раздел <<Невредные советы>> скорее направлен на сбережение времени не преподавателя, но студента,
хотя в определённых случаях представляется целесообразным настаивать на их соблюдении.

В заключение отметим, что исходный код данного пособия размещён в git-репозитории (данные на первой странице),
а это означает, что его можно легко доработать под нужды конкретного педагога. Желаем читателю удачной работы :-)



