\begin{table}[ph]

\begin{tabular}{|l|l|l|}
\hline
			Паскаль
&			Javascript
&			Пояснение
\\\hline
			begin
&
			\{
&
	\multirow{2}{*}{
			Открывающая и закрывающая операторные скобки соответственно.
			Позволяют группировать другие операторы в циклах и т. д.
	}
\\\hline
			end
&			\}
&
\\\hline
			$\begin{array}{l}
			\mbox{var}
			\\
			~~a,b: integer;
			\\
			~~s: string;
			\\
			~~m: array[1..7] of char;
			\end{array}$
&
			{\barecodesnippet{code-snippets/analog-var}{language=C++}}
			var a,b,s;<br/>
			var m=[];
&			В Javascript переменные не имеют фиксированного типа, он определяется автоматически.
			Более того, массив не имеет фиксированного размера иможет состоять из элементов различного типа:
			<br/>
			var m=['a',1,2.3,[4,5]];
			<br/>
			(хотя так обычно не делают).
\\\hline
			\{комментарий\}
&
			/*комментарий*/

			//комментарий
&
			Первый вариант называется многострочным комментарием, второй - однострочным
			(всё, что после // и до конца строки - комментарий,
			то есть компьютером не читается и нужно лишь для удобства человека)
\\\hline
			if(a=b) then begin

			~~c:=0;

			end else begin

			~~d:=1;

			end;
&			if(a==b)\{

			~~c=0;<br/>
			\}else\{<br/>
			~~d=1;<br/>
			\}
&			Условный оператор (ветвление).
			Обратите внимание на то, что для сравнения переменных в условии в JS используется запись
			a==b, а для присваивания - запись c=0
\\\hline
			for i:=1 to n do begin<br/>
			~~s:=s+i;<br/>
			end;
&			for(i=1;i<n;i++)\{<br/>
			~~s+=i;<br/>
			\}
&			Цикл со счётчиком.
			Обратите внимание на сокращённое присваивание:

				i++ значит i=i+1

			s+=i значит s=s+i
\\\hline
			while i<n do begin

			~~s:=s+i;

			~~i:=i+1;

			end;
&			while(i<n)\{<br/>
			~~s+=i;<br/>
			~~i++;<br/>

			\}
&			Цикл с условием (с предусловием).
			Делает то же самое, что и предыдущий, кроме присваивания i начального значения.

\\\hline
\end{tabular}

\end{table}
