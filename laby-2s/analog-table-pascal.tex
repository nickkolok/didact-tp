<table class="table table-bordered normtabl"><tbody>
	<tr>
		<th>
			Паскаль
&			Javascript
&			Пояснение
\\			begin
&			{
		</td>
		<td rowspan="2">
			Открывающая и закрывающая операторные скобки соответственно.
			Позволяют группировать другие операторы в циклах и т. д.
\\			end
&			}
\\			var<br/>
			a,b: integer;<br/>
			s: string;<br/>
			m: array[1..7]of char;
			
&			var a,b,s;<br/>
			var m=[];
&			В Javascript переменные не имеют фиксированного типа, он определяется автоматически.
			Более того, массив не имеет фиксированного размера иможет состоять из элементов различного типа:
			<br/>
			var m=['a',1,2.3,[4,5]];
			<br/>
			(хотя так обычно не делают).
\\			{комментарий}
&			/*комментарий*/
			<br/>
			//комментарий
&			Первый вариант называется многострочным комментарием, второй - однострочным
			(всё, что после // и до конца строки - комментарий,
			то есть компьютером не читается и нужно лишь для удобства человека)
\\			if(a=b) then begin<br/>
			&nbsp;&nbsp;c:=0;<br/>
			end else begin<br/>
			&nbsp;&nbsp;d:=1;<br/>
			end;
&			if(a==b){<br/>
			&nbsp;&nbsp;c=0;<br/>
			}else{<br/>
			&nbsp;&nbsp;d=1;<br/>
			}
	&			Условный оператор (ветвление).
			Обратите внимание на то, что для сравнения переменных в условии в JS используется запись
			a==b, а для присваивания - запись c=0
\\			for i:=1 to n do begin<br/>
			&nbsp;&nbsp;s:=s+i;<br/>
			end;
&			for(i=1;i&lt;n;i++){<br/>
			&nbsp;&nbsp;s+=i;<br/>
		}
&			Цикл со счётчиком.
			Обратите внимание на сокращённое присваивание:
			<br/>
				i++ значит i=i+1 
			<br/>
			s+=i значит s=s+i
\\			while i&lt;n do begin<br/>
			&nbsp;&nbsp;s:=s+i;<br/>
			&nbsp;&nbsp;i:=i+1;<br/>
			end;
&			while(i&lt;n){<br/>
			&nbsp;&nbsp;s+=i;<br/>
			&nbsp;&nbsp;i++;<br/>	
			}
&			Цикл с условием (с предусловием).
			Делает то же самое, что и предыдущий, кроме присваивания i начального значения.
		</td>
	</tr>
</tbody></table>

