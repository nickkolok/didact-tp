Правила оформления программного кода (Code Style, соглашение о стиле кода, стилевой регламент) --- это соглашение о том, как оформлять код в рамках проекта или нескольких проектов.

Казалось бы, что необходимость думать не только о логике работы программы, но и о её оформлении в соответствии с некими правилами должна мешать программисту и повышать трудозатраты на написание кода, но на деле это не так.

Во-первых, привыкнуть к выполнению нескольких простых правил --- это несложно и быстро.

Во-вторых, единообразное оформление значительно облегчает восприятие кода, что особенно важно в тех случаях,
когда над проектом работает несколько человек.
Обычно в таких случаях говорят: <<Код проекта должен выглядеть так, как будто его писал один человек>>.
Кроме того, в реальной работе программист больше читает код, чем пишет его, и больше читает чужой код, чем свой.

В-третьих, наличие правил позволяет программисту не задумываться над тем, как лучше оформить код в данном случае, и это экономит время и силы.
Это вообще достаточно общий принцип: чем более сильные условия наложены, тем лучше результат;
студенты математических направлений не раз наблюдали его в действии на примере многочисленных теорем.

В-четвёртых, это в некоторой мере защищает код программы от бессмысленных изменений (например, от случайно поставленного пробела в конце строки).
Отсутствие бессмысленных (и зачастую не замечаемых человеком) изменений очень важно для корректной работы систем контроля версий (сокращённо СКВ; классический пример СКВ --- git), так как они, как правило, отслеживают изменения построчно.
Code Style борется с ситуациями, когда в <<журнал>> (репозиторий) СКВ записываются изменения, которые ничего не значат.

Итак, при выполнении лабораторных работ студентам следует руководствоваться нижеследующим Code Style.
Заметим, что он в целом значительно мягче большинства регламентов, принятых в крупных проектах,
и может в некоторых случаях дополняться преподавателем.
%Особое внимание следует ещё раз обратить на то, что Code Style различных проектов может радикально отличаться (т. е. некоторые пункты различных Code Style могут быть взаимоисключающими).


\begin{enumerate}
	\item
		Отступы.

		Отступы в коде необходимы, так как существенно облегчают восприятие структуры программы.
		\begin{enumerate}
			\item
				Отступы делаются клавишей TAB.
			\item
				Отступ у первой строки файла отсутствует (аксиома нормировки).
			\item
				При расстановке отступов пустые строки не учитываются.
			\item
				Для изменения отступа у $n$-й строки по сравнению с предыдущей ($(n-1)$-й) должны быть веские причины.
			\item
				Если $n$-я строка заканчивается любой открывающей скобкой $($, $[$ или $\{$,
				то $(n+1)$-я строка смещается на один отступ вправо.
			\item
				Если $n$-я строка начинается любой закрывающей скобкой $)$, $]$ или $\}$,
				то она смещается на один отступ влево.

				При выполнении этих правил строка, начинающаяся с некоторой закрывающей скобки, оказывается ровно под строкой,
				которая закончилась парной открывающей, а всё, что между ними --- как бы внутри.

				Правильно:
				\codesnippet{code-style-snippets/tabs-right-1}
				Неправильно:
				\codesnippet{code-style-snippets/tabs-wrong-1}


			\item
				Внутри switch операторы case/default и break выполняют роль скобок.
				После последнего варианта ``скобка'' пропускается, отступ уменьшается на 2.
				break и continue внутри циклов скобками не считаются.
				Правильно:
				\codesnippet{code-style-snippets/case-right-1}
				Неправильно:
				\codesnippet{code-style-snippets/case-wrong-1}


			\item
				Если открывающая скобка стоит в конце строки, то парная ей должна стоять в начале (и уменьшать отступ).
				Правильно:
				\codesnippet{code-style-snippets/braces-right-1}

				Неправильно:
				\codesnippet{code-style-snippets/braces-wrong-1}

			\item
				Перенос выражений (разрыв строки без точки с запятой) осуществляется по следующим правилам:
				\begin{enumerate}
					\item
						Вторая и последующие части разрываемой строки имеют отступ, увеличенный на 1 по сравнению с первой частью.
					\item
						Бинарные операторы  <\!< и >\!> и префиксные унарные операторы пишутся в начале новой строки.

						Правильно:
						\codesnippet{code-style-snippets/linebreak-stream-right-1}
						Неправильно:
						\codesnippet{code-style-snippets/linebreak-stream-wrong-1}

					\item
						Бинарные операторы + - * / || \&\& и постфиксные унарные операторы пишутся в конце переносимой строки.

						Правильно:
						\codesnippet{code-style-snippets/linebreak-binary-right-1}
						Неправильно:
						\codesnippet{code-style-snippets/linebreak-binary-wrong-1}

					\item
						Операторы + * || и \&\& могут писаться на отдельной строке,
						если разрываемая строка содержит 3 и более однотипных выражения.
						При этом допускается отступ у строки с оператором уменьшить до отступа первой части переносимой строки

						Правильно:
						\codesnippet{code-style-snippets/linebreak-binary-more-than-two-right-1}
						Неправильно:
						\codesnippet{code-style-snippets/linebreak-binary-more-than-two-wrong-1}

				\end{enumerate}


			%\item

		\end{enumerate}
	\item
		Переменные и функции
		\begin{enumerate}
			\item
				Названия переменных и функций по возможности должны иметь осмысленные, адекватные названия.
				Названия переменных типа \textbf{a1}, \textbf{b} допускаются только в случаях,
				когда либо название переменной обусловлено условием задачи (<<С клавиатуры вводятся числа a и b...>>),
				либо использование переменной носит очень эпизодический характер (например, классический счётчик \textbf{i} в цикле).
			\item
				Переменную следует объявлять ровно в том блоке видимости, в котором она используется (не в более внешнем!).
			\item
				Переменные следует объявлять либо в начале блока видимости
				(т.~е. блок начинается с объявления всех переменных, используемых в этом блоке, с соблюдением предыдущего правила),
				либо непосредственно перед использованием.
			\item
				Если есть возможность выделить повторяющиеся или почти повторяющиеся (с небольшими отличиями, например, имя переменной)
				фрагменты кода в функцию --- как правило, это нужно делать.
			\item
				Следует по возможности разделять функции, отвечающие за обработку данных,
				и функции, отвечающие за ввод-вывод.
		\end{enumerate}
\end{enumerate}






