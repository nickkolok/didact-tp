Правила оформления программного кода (Code Style, соглашение о стиле кода, стилевой регламент) --- это соглашение о том, как оформлять код в рамках проекта или нескольких проектов.

Казалось бы, что необходимость думать не только о логике работы программы, но и об её оформлении в соответствии с некими правилами должна мешать программисту и повышать трудозатраты на написание кода, но на деле это не так.

Во-первых, привыкнуть к выполнению нескольких простых правил --- это несложно и быстро.

Во-вторых, единообразное оформление значительно облегчает восприятие кода, что особенно важно в тех случаях,
когда над проектом работает несколько человек.
Обычно в таких случаях говорят: <<Код проекта должен выглядеть так, как будто его писал один человек>>.

В-третьих, наличие правил позволяет программисту не задумываться над тем, как лучше оформить код в данном случае, и это экономит время и силы.
Это вообще достаточно общий принцип: чем более сильные условия наложены, тем лучше результат;
студенты математических направлений не раз наблюдали его в действии на примере многочисленных теорем.

В-четвёртых, это в некоторой мере защищает код программы от бессмысленных изменений (например, от случайно поставленного пробела в конце строки).
Отсутствие бессмысленных (и зачастую не замечаемых человеком) изменений очень важно для корректной работы систем контроля версий (сокращённо СКВ; классический пример СКВ --- git), так как они, как правило, отслеживают изменения построчно.
Code Style борется с ситуациями, когда в <<журнал>> (репозиторий) СКВ записываются изменения, которые ничего не значат.

Итак, при выполнении лабораторных работ студентам следует руководствоваться нижеследующим Code Style.
Заметим, что он значительно мягче большинства регламентов, принятых в крупных проектах,
и может в некоторых случаях дополняться преподавателем.


\begin{enumerate}
	\item
		Отступы.

		Отступы в коде необходимы, так как существенно облегчают восприятие структуры программы.
		\begin{enumerate}
			\item
				Отступы делаются клавишей TAB.
			\item
				Отступ у первой строки файла отсутствует.
			\item
				При расстановке отступов пустые строки не учитываются.
			\item
				Для изменения отступа у $n$-й строки по сравнению с предыдущей ($(n-1)$-й) должны быть веские причины.
			\item
				Если $n$-я строка заканчивается любой открывающей скобкой $($, $[$ или $\{$,
				то $(n+1)$-я строка смещается на один отступ вправо.
			\item
				Если $n$-я строка начинается любой закрывающей скобкой $)$, $]$ или $\}$,
				то она смещается на один отступ влево.

				При выполнении этих правил строка, начинающаяся с некоторой закрывающей скобки, оказывается ровно под строкой,
				которая закончилась парной открывающей, а всё, что между ними --- как бы внутри.

				Правильно:
				\codesnippet{tabs-right1}
%				\begin{lstlisting}{Language={C++},caption={},frame=single,numbers=left}
\shortlisting{
cin>>n;
if(n<0){
	cout<<"n не должно быть отрицательным, используется абсолютная величина"<<endl;
	n=-n;
}
cout<<sqrt(n);
}
%				\end{lstlisting}
		\end{enumerate}
	\item
\end{enumerate}






