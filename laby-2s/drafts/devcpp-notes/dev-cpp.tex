Совет: если по F11 ничего не происходит, возможно, Ваша переносная ЭВМ (ноутбук) предпринимает по F11 какие-либо самостоятельные действия, например, изменяет яркость, включает/выключает Wi-Fi и т. д.

%TODO: иллюстрация

Совет: не используйте русские буквы и пробелы в имени файла и в пути к нему; в частности, не держите файлы на рабочем столе.
Рекомендуется создать папку, например, D:\\TP_Labs, и сохранять файлы в ней.
Несоблюдение этого простого правила поначалу может не иметь выраженного эффекта, но затем неожиданно приводить ко всевозможным отклонениям от нормальной работы программы.
В частности, авторам известен случай, когда Dev C++ внезапно прекращал регировать на внесение изменений в исходный код программы, в названии файла которой была кириллица.
