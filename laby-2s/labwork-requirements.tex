Данный раздел содержить список требований, необходимых для того, чтобы Ваша лабораторная работа была успешно рассмотрена.

\begin{enumerate}
	\item
		Требования к содержанию лабораторной работы, отправляемой на проверку.
	\begin{enumerate}
		\item
			Запрещается сдавать отчёт, в который не входит ничего, кроме титульного листа и/или формулировок заданий.
		\item
			Запрещается повторно сдавать уже проверенный преподавателем отчёт.			
	\end{enumerate}		

\end{enumerate}

\begin{enumerate}
	\item
		Требования к оформлению лабораторной работы, отправляемой на проверку.
	\begin{enumerate}
		\item
			Отчёт по лабораторной работе отправляется в виде одного файла в формате ГОСТ Р ИСО/МЭК 26300-2010 (ODT, LibreOffice: libreoffice.org).
			Файл имеет имя вида \textbf{lab2.1\_Avdeev\_003.odt},
			где вместо \textbf{2} --- номер семестра, вместо \textbf{1} --- номер лабораторной работы в семестре,
			вместо \textbf{Avdeev} --- Ваша фамилия в именительном падеже транслитом,
			вместо \textbf{003} --- номер попытки сдачи лабораторной работы.
		\item
			Файлы с программным кодом, отправляемые на рассмотрение, имеют имя вида \textbf{lab2.1-5\_Avdeev\_003.cpp},
			где вместо \textbf{5} --- номер задания, код к которому содержится в файле.
			Файл должен иметь кодировку UTF-8.
	\end{enumerate}		

\end{enumerate}


