\documentclass[a4paper,14pt]{report} %размер бумаги устанавливаем А4, шрифт 12пунктов
\usepackage[T2A]{fontenc}
\usepackage[utf8]{inputenc}
\usepackage[english,russian]{babel} %используем русский и английский языки с переносами
\usepackage{amssymb,amsfonts,amsmath,mathtext,cite,enumerate,float,amsthm} %подключаем нужные пакеты расширений
\usepackage[pdftex,unicode,colorlinks=true,linkcolor=blue]{hyperref}
\usepackage{indentfirst} % включить отступ у первого абзаца
\usepackage[dvips]{graphicx} %хотим вставлять рисунки?
\graphicspath{{illustr/}}%путь к рисункам

\makeatletter
\renewcommand{\@biblabel}[1]{#1.} % Заменяем библиографию с квадратных скобок на точку:
\makeatother %Смысл этих трёх строчек мне непонятен, но поверим "Запискам дебианщика"

\usepackage{geometry} % Меняем поля страницы. 
\geometry{left=1cm}% левое поле
\geometry{right=1cm}% правое поле
\geometry{top=1cm}% верхнее поле
\geometry{bottom=2cm}% нижнее поле

\renewcommand{\theenumi}{\arabic{enumi}}% Меняем везде перечисления на цифра.цифра
\renewcommand{\labelenumi}{\arabic{enumi}}% Меняем везде перечисления на цифра.цифра
\renewcommand{\theenumii}{.\arabic{enumii}}% Меняем везде перечисления на цифра.цифра
\renewcommand{\labelenumii}{\arabic{enumi}.\arabic{enumii}.}% Меняем везде перечисления на цифра.цифра
\renewcommand{\theenumiii}{.\arabic{enumiii}}% Меняем везде перечисления на цифра.цифра
\renewcommand{\labelenumiii}{\arabic{enumi}.\arabic{enumii}.\arabic{enumiii}.}% Меняем везде перечисления на цифра.цифра

% Пакет для отображения исходного кода - с http://www.inp.nsk.su/~baldin/LaTeX/lurs-code.pdf
\usepackage{listings}
%\usepackage{listingsutf8}
% подгружаемые языки — подробнее в документации listings
\lstloadlanguages{C++}
% Конфигурируем
\lstset{
	language=C++, % выбираем язык по умолчанию
	frame=single, % рамка
	commentstyle=\itshape\textcolor[rgb]{0.5,0.5,0.5}, % шрифт для комментариев
	stringstyle=\bfseries, % шрифт для строк
	numbers=left,              % где поставить нумерацию строк (слева\справа)
	numberstyle=\tiny,         % размер шрифта для номеров строк
	tabsize=2,                 % размер табуляции по умолчанию равен 2 пробелам
}

% А эта тёмная магия позволяет нормально работать с кириллицей в листингах
\input{../lib/listingcyr}


\LARGE
\begin{document}
\input{../lib/macro}
\newcounter{labworkcounter}
\setcounter{labworkcounter}{1}

\newcounter{labtaskcounter}[labworkcounter]

\newcommand{\labwork}[1]{
	\newpage
	\phantomsection
	\addcontentsline{toc}{section}{{\thelabworkcounter}. #1}
	\section*{Лабораторная работа №{\thelabworkcounter}}
	\textit{Тема: <<{#1}>>}
	%\vspace{10mm}
	\stepcounter{labworkcounter}
}

\newcommand{\labtask}{
	\stepcounter{labtaskcounter}%Чтобы не с нуля начинать
	\subsection*{Задание \thelabtaskcounter}
}

\newcommand{\reservedtasks}{
	\subsection*{Резервные задания}
}

\newcommand{\typerrors}{
	\subsection*{Типовые ошибки}
}


\newcommand{\labworkquestions}{
	\subsection*{Вопросы к работе}
}

\newcommand{\codeexample}[2]{
	\phantomsection
	\addtocounter{lstlisting}{1}
	\addcontentsline{toc}{section}{\thelstlisting. #1}
	\addtocounter{lstlisting}{-1}
	\lstinputlisting[label=#2,caption={#1}]{cpp-examples/#2.cpp}
}

\newcommand{\codesnippet}[1]{
	\lstinputlisting[caption={}]{#1.cpp}
}

\newcommand{\barecodesnippet}[1]{
	\lstinputlisting[caption={},numbers=none,frame=none]{#1.cpp}
}


\newtheorem{typerrorthm}{Типовая ошибка}
\newenvironment{typerror}[1][]{
	\begin{typerrorthm} #1\ifthenelse{\equal{#1}{}}{}{.}
	\addcontentsline{toc}{section}{{\thetyperrorthm}. #1}
	\par
}{
	\end{typerrorthm}
}


Исходные тексты (в том числе файлы с кодом) доступны по адресу https://github.com/nickkolok/didact-tp/

Хэш-ревизия сборки (хэш последнего коммита git):
\input{.gitrevision}

Дата и время сборки:
\input{.builddate}

\tableofcontents

\addcontentsline{toc}{chapter}{Лабораторные работы}

Лабораторная работа №1

Тема: <<Работа со строками как с массивами символов>>



Задание 1.

Пользователем вводится строка (возможно, содержащая пробелы).
Произвести над ней заданные операции.

Указание. Словом считаеся последовательность малых или больших латинских букв A-Z, a-z.

\begin{enumerate}
	\item
		Подсчитать количество гласных букв (a,e,o,o,u,y).
	\item
		Подсчитать количество биграмм <<ab>>.
	\item
		Инвертировать регистр букв.
	\item
		Превратить большие согласные буквы в маленькие.
	\item
		Удалить из строки все html-тэги, т. е. подстроки, начинающиеся с < и заканчивающиеся ближайшей > либо концом строки.
	\item
		Иногда при наборе текста в начале слова набирающий не успевает вовремя отпустить Shift, и получается нечто вроде <<TExt>>.
		Исправить все слова, начинающиеся со сдвоенной согласной буквы.
	\item
		Иногда при наборе текста набирающий, желая поставить многоточие, забывает поставить третью точку в нём, получая нечто вроде <<Text..>>.
		Исправить такие ситуации.
	\item
		Эмоциональные школьники зачастую пишут много восклицательных знаков подряд.
		Везде, где количество подряд идущих восклицательных знаков превышает три, остальные удалить.
	\item
		Перевернуть строку.
	\item
		Заменить все вхождения подстроки 'ck' на 'kk'.
	\item
		По типографским правилам набора между запятой и предшествующим словом пробел не ставится.
		Найти все такие лишние пробелы и убрать их.
	\item
		По типографским правилам набора после запятой ставится пробел.
		Расставить недостающие пробелы.		
	\item
		Вычислить длину наибольшего фрагмента текста, заключённого между запятыми.
	\item
		Проверить, есть ли в строке закрывающая скобка <<)>>, идущая раньше, чем первая из открывающих.
	\item
		Проверить, что количество открывающих скобок <<(>> в строке соответствует количеству закрывающих <<)>>.

\end{enumerate}

Задание 2.

Программу, написанную в задании 1, видоизменить так, чтобы она содержала функцию, обрабатывающую строку.
Название функции должно быть адекватным.
Функция не должна самостоятельно выводить что-либо на экран.
Функция не должна изменять переданную строку.

Указание. При необходимости для копирования строк воспользуйтесь функцией strcpy.

Задание 3.

Покрыть функцию, написанную в задании 2, тестами,
т. е. вставить в программу код, перед началом её выполнения убеждающийся в правильности выдаваемых написанной функцией результатов
на некоторых специально подобранных характерных примерах исходных данных,
для которых требуемый результат известен.

Задание 4.

Измерить среднее время выполнения набора тестов, составленных в задании 3, с помощью их многократного циклического повторения.

Указание. Пример измерения времени дан в листинге \ref{timecounter}




Резервные задания.
\begin{enumerate}
	\item
		Вычислить длину наибольшего фрагмента текста, заключённого между запятыми и не содержащего знаков конца предложения.
	\item
		Подсчитать, сколько слов написано ЗаБоРчИкОм
	\item
		Выяснить, является ли строка <<перевёртышем>>, например, как <<Аргентина манит негра>>.
		Регистр букв и небуквенные символы не учитывать.
\end{enumerate}


Задание 5.

Используя многократное (в цикле) выполнение тестов, написанных в задании 3, определить, какой прирост производительности даёт вызов функции strlen не при проверке условия завершения цикла, а до начала выполнения цикла с сохранением значения в переменную.




\labwork{Передача строк в функции и возвращение строк из функций}

Задание 1.

Составьте программу, которая запрашивает у пользователя натуральное число, а затем выводит это число и согласованное с ним слово, соответствующее Вашему варианту.

Например, если Ваше слово <<гриб>>, а пользователь ввёл число 55, программа должна вывести сообщение <<55 грибов>>, а если пользователь ввёл число 21, то <<21 гриб>>.

Указание. Использовать деление с остатком, конструкции if-else и/или switch-case.

Слова:

\begin{enumerate}

\item стол

\item окно

\item  герань

\item  земля

\item  стул

\item  роза 

\item  звезда

\item  яблоко

\item  рулон

\item  гора

\item  башня

\item  окно

\item  абрикос

\item  картина

\item  провод



\end{enumerate}


Задание 2.

Выделить существенный код в функцию, принимающую число и необходимое количество вариантов строковой переменной и возвращающую требуемую строковую переменную.

Задание 3.

Покрыть юнит-тестами функцию, написанную Вами при выполнении задания 2, измерить время её выполнения.

Задание 4.

Используя написанную в задании 2 функцию, вывести аналогичные фразы для слов.

\begin{enumerate}

\item 	тетрадь, карандаш

\item 	стена, кирпич

\item 	дерево, лист

\item 	учитель, ученик

\item 	дом, окно

\item 	машина, колесо

\item 	мать, сын

\item 	человек, жизнь

\item 	закон, врач

\item 	песня, слово

\item 	сказка, ложь

\item 	 свет, частица

\item 	снег, лопата

\item 	война, кровь

\item 	дым, свет


\end{enumerate}

Пример работы программы:

Введите число:

5

5 грибов

5 оленей

5 лисиц  


\labwork{Работа с указателями}

\labtask

Заведите целочисленную переменную $a$, получите указатель на неё и далее обращайтесь к ней только через указатель.
Заведите динамический указатель и выделите новый участок памяти для хранения числа $d$.
Запросите у пользователя числа $a$ и $d$.
Выведите значение выражения, соответствующего Вашему варианту.

\begin{enumerate}

\item $a^3-2ad$

\item $3a^2-5d$

\item  $7a-d^3$

\item  $4ad^2+2a^3$

\item  $3ad+7d^2$

\item  $1-d^3+a^2$

\item  $4d^4+a^2$

\item  $5d-3a^4$

\item  $2a^2+ad-1$

\item  $6a^2d-2d^2$

\item  $3ad^2-7a^2$

\item  $a^4+3ad^2$

\item  $a^3d-2a$

\item  $a^2d+6ad^2$

\item  $5a+7d^3$

\end{enumerate}


\labtask

Выполните указанные вычисления.
Значения переменных, стоящих в правой части формулы, вводятся пользователем в том порядке, в котором эти переменные встречаются в формуле, слева направо, сверху вниз.
Для сохранения значений в переменную используйте прямое обращение к переменной, для получения значений переменной - обращение через указатель.


\begin{enumerate}

\item $\rho =\frac{p}{gh}$

\item $a=\frac{v^2}{r}$

\item  $s=v_0t+\frac{at^2}{2}$

\item  $p=\frac{mg}{S}$

\item  $\omega =\frac{2\pi }{T}$

\item  $n=\frac{N}{V}$

\item  $\eta =1-\frac{T_2}{T_1}$

\item  $\varphi =k\frac{Q}{r}$

\item  $C=\frac{\epsilon \epsilon_0S}{d}$

\item  $F=\frac{kQ_1}{\epsilon r^2}$

\item  $P=\frac{A}{t}$

\item  $r=\frac{mv}{qB}$

\item  $R=\frac{R_1R_2}{R_1+R_2}$

\item  $I=\frac{U}{R}$

\item  $n=\frac{n_2}{n_1}$

\end{enumerate}

\labworkquestions

\begin{enumerate}

	\item Что хранится в переменной типа {\bf  «указатель»}?

	\item Дан фрагмент кода

	{\bf int*** mas=new int**[7];}

	Что происходит в этой строке?
	Какой тип имеет выражение  {\bf mas[2] }?

	\item Для чего используется ключевое слово  {\bf new }?

	\item Какой тип имеет выражение  {\bf new double*[16] }?

	\item  Когда при использовании указателя не требуется вызов оператора {\bf delete} ?

	\item  Для чего предназначен оператор  {\bf delete} ?

	\item Дан фрагмент программы

		    {\bf
		       int k=0;

		       int* pk=\&k;

		       k=5;

		       cout << (*pk);
				}

	Что будет выведено на экран и почему?

\end{enumerate}


\labwork{Работа с одномерными массивами}

\labtask

С клавиатуры вводится длина целочисленного массива.
Затем пользователь выбирает, ввести массив с клавиатуры или сгенерировать случайным образом.

\begin{enumerate}

	\item Найти минимальный элемент среди всех элементов массива.

	\item Найти максимальный элемент среди всех элементов массива.

	\item Найти минимум модулей всех элементов массива.

	\item Найти максимум модулей всех элементов массива.

	\item Найти минимум модулей разностей всех соседних элементов массива.

	\item  Найти максимум модулей разностей всех соседних элементов массива.

	\item Найти разность между максимальным и минимальным элементами массива.

	\item Найти среднее арифметическое всех элементов массива.

	\item Найти среднее арифметическое модулей всех элементов массива.

	\item Найти сумму всех элементов массива.

	\item Найти произведение всех элементов массива.

	\item Найти сумму квадратов всех элементов массива.

	\item Найти сумму кубов всех элементов массива.

	\item Найти произведение всех ненулевых элементов массива.

	\item Найти количество всех ненулевых элементов массива.

\end{enumerate}


\labtask

\begin{enumerate}

	\item Найти количество единиц среди всех элементов массива.

	\item Найти количество максимальных элементов среди всех элементов массива.

	\item Найти количество минимальных элементов среди всех элементов массива.

	\item Найти количество максимальных элементов по модулю среди всех элементов массива.

	\item Найти номер первого максимального элемента среди всех элементов массива.

	\item Найти номер первого минимального элемента среди всех элементов массива.

	\item Найти номер первого максимального элемента по модулю среди всех элементов массива.

	\item Найти номер последнего максимального элемента среди всех элементов массива.

	\item Найти номер последнего минимального элемента среди всех элементов массива.

	\item Найти номер последнего максимального по модулю элемента среди всех элементов массива.

	\item Найти номер последнего минимального по модулю элемента среди всех элементов массива.

	\item Найти номер первого нулевого элемента массива.

	\item Найти номер первого ненулевого элемента массива.

	\item Найти номер первого положительного элемента массива.

	\item Найти номер первого отрицательного элемента массива.

\end{enumerate}

\reservedtasks

\begin{enumerate}

	\item Найти номер первого минимального элемента по модулю среди всех элементов массива.

	\item Найти количество минимальных элементов по модулю среди всех элементов массива.

\end{enumerate}


\labtask

Измените программы, написаные в заданиях 1 и 2, выделив существенный код в функции.
Следите за тем, чтобы функции были специализированы: либо работали с потоками ввода-вывод, либо производили вычисления.
Напишите тесты (где это возможно).

Указание. Не требуется писать тесты для функции генерации случайного массива и функции ввода массива пользователем. 

\labtask

Доработать написанную в предыдущем задании программу (унаследованную от задания 1) так, чтобы она наряду с массивом и методом его ввода запрашивала у пользователя границы обрабатываемого подмассива (могла обрабатывать не весь массив целиком, а некоторую его часть), т. е. индексы (номера) элемента, с которого начинать обработку, и индекс элемента, на котором закончить обработку.

\labworkquestions

\begin{enumerate}

	\item
		Как передать массив в функцию?
	\item
		В каких случаях для передачи массива в функцию недостаточно передать указатель?
	\item
		Как вернуть массив из функции? Достаточно ли вернуть указатель?
	\item
		Как передать в функцию массив, полученный из данного отбрасыванием $n$ первых элементов?
\end{enumerate}




\labwork{Работа с одномерными массивами, часть 2}

\labtask

С клавиатуры вводится длина целочисленного массива.
Сгенерировать целочисленный массив указанной длины, заполнив его случайными числами в диапазоне от -20 до 20 включительно.
Сгенерированный массив вывести на экран.
После этого на основе полученного массива сформировать новый массив в соответствии с номером Вашего варианта.
Новый массив вывести на экран.

Указание. При выполнении работы разрешается использовать и модифицировать функции, написанные в лабораторной работе №4.

\begin{enumerate}

	\item
		Из элементов, отличающихся от минимального не более, чем на 2.

	\item
		Из элементов, отличающихся от максимального не менее, чем на 3.

	\item
		Из элементов, отличающихся от минимального по модулю элемента более, чем на 5.

	\item
		Из элементов, отличающихся от максимального по модулю элемента менее, чем на 4.

	\item
		Из номеров тех элементов, которые не больше, чем  минимум модулей разностей всех соседних элементов массива.

	\item
		Из номеров тех элементов, которые не меньше, чем максимум модулей разностей всех соседних элементов массива.

	\item
		Из номеров тех элементов, которые меньше, чем разность между максимальным и минимальным элементами массива.

	\item
		Из номеров тех элементов, которые больше, чем среднее арифметическое всех элементов массива.

	\item
		Из элементов, которые больше, чем среднее арифметическое модулей всех элементов массива.

	\item
		Из тех элементов, которые больше, чем последняя цифра суммы всех элементов массива.

	\item
		Из номеров тех элементов, которые меньше, чем последняя цифра произведения всех элементов массива.

	\item
		Из тех элементов, которые больше, чем последняя цифра суммы квадратов всех элементов массива.

	\item
		Из номеров тех элементов, которые меньше, чем последняя цифра суммы кубов всех элементов массива.

	\item
		Из номеров тех элементов, которые меньше, чем произведение всех ненулевых элементов массива.

	\item
		Из тех элементов, которые больше, чем количество всех ненулевых элементов массива.

\end{enumerate}


\labtask

Выделите существенный код в функции, напишите тесты (где это возможно).

Указание. Не требуется писать тесты для функции генерации случайного массива и функции ввода массива пользователем. 

Указание. При возвращении массивов из функции придерживаться соглашения: нулевой элемент массива хранит его длину.
При передаче массива в функцию придерживаться соглашения: размер массива передаётся отдельно от самого массива.
При определении индексов элементов элемент, хранящий длину, не учитывать, изменить указатель с помощью арифметики указателей.

\labworkquestions

\begin{enumerate}

	\item
		Как создать массив требуемой длины?
	\item
		Как сгенерировать случайное число в указанном диапазоне?
	\item
		Как объявить функцию, возвращающую массив значений типа \textbf{double}?
	\item
		Как инициализировать массив?
	\item
		Почему иногда бывает удобно при возвращении массивов из функции придерживаться соглашения: нулевой элемент массива хранит его длину?
		Как работать с таким <<надставленным>> массивом впоследствии?
\end{enumerate}




\labwork{Рекурсивная обработка массивов. Измерение времени работы функции}

\labtask

На основе каждой из функций обработки массива, написанных Вами в лабораторной работе №4, составьте по две функции:
одну --- обрабатывающую массив в цикле, другую --- обрабатывающую массив рекурсивно.
С помощью тестов, написанных Вами в той же лабораторной работе, убедитесь в корректной работе всех четырёх функций.

Указание. Пример рекурсивной обработки массива дан в листинге \ref{minimum-even-recursive}.

\labtask

С помощью функции генерации случайного массива, написанной Вами ранее, сгенерируйте 10240 случайных массивов длиной 1024.
С помощью цикла измерьте среднее время работы каждой из функций.
Выведите на экран сумму всех возвращённых функциями значений.

Указание. Пример измерения времени дан в листинге \ref{timecount}.

Сделайте вывод о соотношении времени работы рекурсивной и циклической функций, включите его в отчёт.
Запустите программу несколько раз.
Сделайте вывод о стабильности или нестабильности этого отношения и абсолютных величин затрачиваемого времени, включите его в отчёт.
Укажите причины такой стабильности или нестабильности для абсолютного времени и для отношения.

\labtask

Предполагая, что длина массива не превосходит 120 элементов, напишите несколько функций, использующих в качестве размера массива и итератора цикла переменные различных известных Вам типов (тип размера и тип итератора должны совпадать).
Измерьте соотношение времени работы функций и абсолютное время работы функций.

Сделайте вывод о наиболее быстрой функции, включите его в отчёт.

\labtask

Выполните то же, но для массивов из 8 элементов.

\labworkquestions
\begin{enumerate}
	\item
		Что такое рекурсия?
	\item
		Чем прямая рекурсия отличается от косвенной?
	\item
		Какие виды циклов существуют в языке С++?
	\item
		Может ли размер массива быть отрицательным?
	\item
		Как измерить время работы программы?
	\item
		Какой заголовочный файл следует подключить для измерения времени работы программы или её части?
	\item
		За что отвечает константа \textbf{CLOCKS\_PER\_SEC}?
		Чему она равна на используемой связке компилятор+ОС?
\end{enumerate}



\newpage
\phantomsection
\addcontentsline{toc}{chapter}{Правила оформления программного кода}
\chapter*{Правила оформления программного кода}
Правила оформления программного кода (Code Style, соглашение о стиле кода, стилевой регламент) --- это соглашение о том, как оформлять код в рамках проекта или нескольких проектов.

Казалось бы, что необходимость думать не только о логике работы программы, но и об её оформлении в соответствии с некими правилами должна мешать программисту и повышать трудозатраты на написание кода, но на деле это не так.

Во-первых, привыкнуть к выполнению нескольких простых правил --- это несложно и быстро.

Во-вторых, единообразное оформление значительно облегчает восприятие кода, что особенно важно в тех случаях,
когда над проектом работает несколько человек.
Обычно в таких случаях говорят: <<Код проекта должен выглядеть так, как будто его писал один человек>>.

В-третьих, наличие правил позволяет программисту не задумываться над тем, как лучше оформить код в данном случае, и это экономит время и силы.
Это вообще достаточно общий принцип: чем более сильные условия наложены, тем лучше результат;
студенты математических направлений не раз наблюдали его в действии на примере многочисленных теорем.

В-четвёртых, это в некоторой мере защищает код программы от бессмысленных изменений (например, от случайно поставленного пробела в конце строки).
Отсутствие бессмысленных (и зачастую не замечаемых человеком) изменений очень важно для корректной работы систем контроля версий (сокращённо СКВ; классический пример СКВ --- git), так как они, как правило, отслеживают изменения построчно.
Code Style борется с ситуациями, когда в <<журнал>> (репозиторий) СКВ записываются изменения, которые ничего не значат.

Итак, при выполнении лабораторных работ студентам следует руководствоваться нижеследующим Code Style.
Заметим, что он значительно мягче большинства регламентов, принятых в крупных проектах,
и может в некоторых случаях дополняться преподавателем.


\begin{enumerate}
	\item
		Отступы.

		Отступы в коде необходимы, так как существенно облегчают восприятие структуры программы.
		\begin{enumerate}
			\item
				Отступы делаются клавишей TAB.
			\item
				Отступ у первой строки файла отсутствует.
			\item
				При расстановке отступов пустые строки не учитываются.
			\item
				Для изменения отступа у $n$-й строки по сравнению с предыдущей ($(n-1)$-й) должны быть веские причины.
			\item
				Если $n$-я строка заканчивается любой открывающей скобкой $($, $[$ или $\{$,
				то $(n+1)$-я строка смещается на один отступ вправо.
			\item
				Если $n$-я строка начинается любой закрывающей скобкой $)$, $]$ или $\}$,
				то она смещается на один отступ влево.

				При выполнении этих правил строка, начинающаяся с некоторой закрывающей скобки, оказывается ровно под строкой,
				которая закончилась парной открывающей, а всё, что между ними --- как бы внутри.

				Правильно:
				\codesnippet{code-style-snippets/tabs-right-1}
%				\begin{lstlisting}{Language={C++},caption={},frame=single,numbers=left}
%\shortlisting{
%cin>>n;
%if(n<0){
%	cout<<"n не должно быть отрицательным, используется абсолютная величина"<<endl;
%	n=-n;
%}
%cout<<sqrt(n);
%}
%				\end{lstlisting}
		\end{enumerate}
	\item
\end{enumerate}









\newpage
\phantomsection
\addcontentsline{toc}{chapter}{Листинги программ}
\chapter*{Листинги программ}

\codeexample{Измерение времени выполнения кода}{timecount}
\newpage
\codeexample{Передача строки в функцию и возвращение строки функцией}{string-char-to-function}
\codeexample{Замена "ck" на "k}{string-char-processing}
\codeexample{Указатели и функции}{functions-and-pointers}
\newpage
\codeexample{Оператор \textbf{sizeof}}{sizeof}
\newpage
\codeexample{Кэширование значений, возвращаемых функцией}{fibo-cache}
\codeexample{Арифметические и логические операции с указателями}{pointers-arithmetic}
\codeexample{Рекурсивная и нерекурсивная обработка массивов}{minimum-even-recursive}

\newpage
\chapter*{Приложение --- типовые ошибки}
\input{typical-errors/signed-instead-of-unsigned}
\begin{typerror}[Некорректное сравнение беззнакового выражения со знаковым]
	\label{TE_signed-with-unsigned-comparison}

	Следует избегать операций сравнения, которых можно избежать, зная, что некоторые переменные неотрицательны.

	Неправильно:
	\codesnippet{typical-errors-snippets/signed-with-unsigned-comparison-wrong}

	Сообщение <<\textbf{Введено отрицательное число, используется модуль}>> никогда не будет выведено.
	
\end{typerror}

\begin{typerror}
	Получение указателя на переменную до её объявления.

	С этой ошибкой программа может не компилироваться.

	Неправильно:
	\codesnippet{typical-errors-snippets/pointer-before-declaration-1-wrong}
	Правильно:
	\codesnippet{typical-errors-snippets/pointer-before-declaration-1-right}

	Вообще не следует использовать переменную выше той строки, в которой она объявлена.
\end{typerror}

\begin{typerror}[Использование оператора побитового исключающего ИЛИ (XOR) \textbf{\^} для возведения в степень]
	\label{TE_xor-as-power}

	С этой ошибкой программа компилируется, но работает неправильно.

	В языке C++ оператор \textbf{\^} предназначен не для возведения в степень, как можно было бы подумать, а для побитового исключающего ИЛИ.
	Кроме того, он имеет более низкий приоритет, чем умножение.

	Неправильно:
	\codesnippet{typical-errors-snippets/xor-as-power-1-wrong}

	Правильно:
	\codesnippet{typical-errors-snippets/xor-as-power-1-right}

	Для использования функции возведения в степень \textbf{pow} требуется подключение заголовочного файла \textbf{<cmath>}.
\end{typerror}




\end{document}
