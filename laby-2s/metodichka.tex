\documentclass[a4paper,14pt]{report} %размер бумаги устанавливаем А4, шрифт 12пунктов
\usepackage[T2A]{fontenc}
\usepackage[utf8]{inputenc}
\usepackage[english,russian]{babel} %используем русский и английский языки с переносами
\usepackage{amssymb,amsfonts,amsmath,mathtext,cite,enumerate,float,amsthm} %подключаем нужные пакеты расширений
\usepackage[pdftex,unicode,colorlinks=true,linkcolor=blue]{hyperref}
\usepackage{indentfirst} % включить отступ у первого абзаца
\usepackage[dvips]{graphicx} %хотим вставлять рисунки?
\graphicspath{{illustr/}}%путь к рисункам

\makeatletter
\renewcommand{\@biblabel}[1]{#1.} % Заменяем библиографию с квадратных скобок на точку:
\makeatother %Смысл этих трёх строчек мне непонятен, но поверим "Запискам дебианщика"

\usepackage{geometry} % Меняем поля страницы. 
\geometry{left=1cm}% левое поле
\geometry{right=1cm}% правое поле
\geometry{top=1cm}% верхнее поле
\geometry{bottom=2cm}% нижнее поле

\renewcommand{\theenumi}{\arabic{enumi}}% Меняем везде перечисления на цифра.цифра
\renewcommand{\labelenumi}{\arabic{enumi}}% Меняем везде перечисления на цифра.цифра
\renewcommand{\theenumii}{.\arabic{enumii}}% Меняем везде перечисления на цифра.цифра
\renewcommand{\labelenumii}{\arabic{enumi}.\arabic{enumii}.}% Меняем везде перечисления на цифра.цифра
\renewcommand{\theenumiii}{.\arabic{enumiii}}% Меняем везде перечисления на цифра.цифра
\renewcommand{\labelenumiii}{\arabic{enumi}.\arabic{enumii}.\arabic{enumiii}.}% Меняем везде перечисления на цифра.цифра

% Пакет для отображения исходного кода - с http://www.inp.nsk.su/~baldin/LaTeX/lurs-code.pdf
\usepackage{listings}
%\usepackage{listingsutf8}
% подгружаемые языки — подробнее в документации listings
\lstloadlanguages{C++}
% Конфигурируем
\lstset{
	language=C++, % выбираем язык по умолчанию
	frame=single, % рамка
	commentstyle=\itshape, % шрифт для комментариев
	stringstyle=\bfseries, % шрифт для строк
	numbers=left,              % где поставить нумерацию строк (слева\справа)
	numberstyle=\tiny,         % размер шрифта для номеров строк
	tabsize=2,                 % размер табуляции по умолчанию равен 2 пробелам
}

% А эта тёмная магия позволяет нормально работать с кириллицей в листингах
% Copyright Nikolay Avdeev aka NickKolok aka Николай Авдеев 2016

% Всем привет из снежного Воронежа! 

% This file is part of LISTINGCYR.

%    LISTINGCYR is free software: you can redistribute it and/or modify
%    it under the terms of the GNU General Public License as published by
%    the Free Software Foundation, either version 3 of the License, or
%    (at your option) any later version.

%    LISTINGCYR is distributed in the hope that it will be useful,
%    but WITHOUT ANY WARRANTY; without even the implied warranty of
%    MERCHANTABILITY or FITNESS FOR A PARTICULAR PURPOSE.  See the
%    GNU General Public License for more details.

%    You should have received a copy of the GNU General Public License
%    along with CHAS-CORRECT.  If not, see <http://www.gnu.org/licenses/>.

%  (Этот файл — часть LISTINGCYR.

%   LISTINGCYR - свободная программа: вы можете перераспространять её и/или
%   изменять её на условиях Стандартной общественной лицензии GNU в том виде,
%   в каком она была опубликована Фондом свободного программного обеспечения;
%   либо версии 3 лицензии, либо (по вашему выбору) любой более поздней
%   версии.

%   CHAS-CORRECT распространяется в надежде, что она будет полезной,
%   но БЕЗО ВСЯКИХ ГАРАНТИЙ; даже без неявной гарантии ТОВАРНОГО ВИДА
%   или ПРИГОДНОСТИ ДЛЯ ОПРЕДЕЛЕННЫХ ЦЕЛЕЙ. Подробнее см. в Стандартной
%   общественной лицензии GNU.

%   Вы должны были получить копию Стандартной общественной лицензии GNU
%   вместе с этой программой. Если это не так, см.
%   <http://www.gnu.org/licenses/>.)
%





% Юзер, помни!
% Сей файл под GNU GPLv3.
% Слинковался - открой сорцы!
% Copyright Nikolay Avdeev 2016
% nickkolok@mail.ru or avdeev@math.vsu.ru

% Пользуясь случаем, передаю привет из Воронежа товарищу @virens

%Спасибо юзеру waverider за http://dxdy.ru/topic18924-15.html
\lstset{literate=
	{А}{{\CYRA}}1
	{Б}{{\CYRB}}1
	{В}{{\CYRV}}1
	{Г}{{\CYRG}}1
	{Д}{{\CYRD}}1
	{Е}{{\CYRE}}1
	{Ё}{{\CYRYO}}1
	{Ж}{{\CYRZH}}1
	{З}{{\CYRZ}}1
	{И}{{\CYRI}}1
	{Й}{{\CYRISHRT}}1
	{К}{{\CYRK}}1
	{Л}{{\CYRL}}1
	{М}{{\CYRM}}1
	{Н}{{\CYRN}}1
	{О}{{\CYRO}}1
	{П}{{\CYRP}}1
	{Р}{{\CYRR}}1
	{С}{{\CYRS}}1
	{Т}{{\CYRT}}1
	{У}{{\CYRU}}1
	{Ф}{{\CYRF}}1
	{Х}{{\CYRH}}1
	{Ц}{{\CYRC}}1
	{Ч}{{\CYRCH}}1
	{Ш}{{\CYRSH}}1
	{Щ}{{\CYRSHCH}}1
	{Ъ}{{\CYRHRDSN}}1
	{Ы}{{\CYRERY}}1
	{Ь}{{\CYRSFTSN}}1
	{Э}{{\CYREREV}}1
	{Ю}{{\CYRYU}}1
	{Я}{{\CYRYA}}1
	{а}{{\cyra}}1
	{б}{{\cyrb}}1
	{в}{{\cyrv}}1
	{г}{{\cyrg}}1
	{д}{{\cyrd}}1
	{е}{{\cyre}}1
	{ё}{{\cyryo}}1
	{ж}{{\cyrzh}}1
	{з}{{\cyrz}}1
	{и}{{\cyri}}1
	{й}{{\cyrishrt}}1
	{к}{{\cyrk}}1
	{л}{{\cyrl}}1
	{м}{{\cyrm}}1
	{н}{{\cyrn}}1
	{о}{{\cyro}}1
	{п}{{\cyrp}}1
	{р}{{\cyrr}}1
	{с}{{\cyrs}}1
	{т}{{\cyrt}}1
	{у}{{\cyru}}1
	{ф}{{\cyrf}}1
	{х}{{\cyrh}}1
	{ц}{{\cyrc}}1
	{ч}{{\cyrch}}1
	{ш}{{\cyrsh}}1
	{щ}{{\cyrshch}}1
	{ъ}{{\cyrhrdsn}}1
	{ы}{{\cyrery}}1
	{ь}{{\cyrsftsn}}1
	{э}{{\cyrerev}}1
	{ю}{{\cyryu}}1
	{я}{{\cyrya}}1
}

% Люди, любите друг друга, используйте Linux и поступайте на матфак ВГУ!






\LARGE
\begin{document}
\newcommand{\pp}{Предположим противное}
%\newcommand{\pp}{{\LARП\!\!\!\!п~}}
\newcommand{\dokvo}{\paragraph{Доказательство.}}
\newcommand{\dokno}{\textbf {Доказано.}}
\newcommand{\neobh}{\paragraph{Необходимость.}}
\newcommand{\dost }{\paragraph{Достаточность.}}
\newcommand{\opred}{\paragraph{Определение.}}
\newcommand{\mnemo}{\paragraph{Мнемоника.}}
\newcommand{\N}{\mathbb{N}}
\newcommand{\Z}{\mathbb{Z}}
\newcommand{\Q}{\mathbb{Q}}
\newcommand{\R}{\mathbb{R}}
\newcommand{\one}[1]{\mathbb{I}_{#1}}
\renewcommand{\C}{\mathbb{C}}
\newcommand{\Beta}{B}%Костыль, а что поделать?
\newcommand{\Rn}{$\mathbb{R}^n~$}
\newcommand{\Rm}{$\mathbb{R}^m~$}
\renewcommand{\epsilon}{\varepsilon}
\renewcommand{\geq}{\geqslant}
\renewcommand{\leq}{\leqslant}
\newcommand{\fXR}{Пусть $X \subset \R, f:X \to \R$ }
\newcommand{\fXRx}{\fXR, $x_0$ - предельная точка $X$ }
\newcommand{\sgn}{\mathrm{sgn}~}
\newcommand{\nid}{\Leftrightarrow}
\newcommand{\intl}{\int\limits}
\newcommand{\suml}{\sum\limits}
\newcommand{\Models}{|\!\!\!=\!\!\!|}
\newcommand{\Rightleftarrow}{\Leftrightarrow}

\newcommand{\xI}{{\vec{\xi}}}%Костыль для тервера, очень уж там часто встречается
\newcommand{\calF}{\mathcal{F}}
\newcommand{\calB}{\mathcal{B}}
\newcommand{\GOFP}{$G \sim \left<\Omega,\calF,P\right>$}

\newenvironment{zamena}[1][c]{=\left<\begin{array}{#1}}{\end{array}\right>=}

\newtheorem{theorem}{Теорема}[section]
\newenvironment{teorema}[1][{}]{\begin{theorem}{#1}\upshape}{\end{theorem}}

\theoremstyle{definition}

\newtheorem{zamech}{Замечание}[section]
\newtheorem{primer}{Пример}[section]
\newtheorem{opr}{Определение}[section]

\newtheorem{sledstvie}{Следствие}[theorem]
\newtheorem{utverzhd}[theorem]{Утверждение}
\newtheorem{lemma}[theorem]{Лемма}

\long\def\comment{}

\newcounter{labworkcounter}
\setcounter{labworkcounter}{1}

\newcounter{labtaskcounter}[labworkcounter]

\newcommand{\labwork}[1]{
	\newpage
	\section*{Лабораторная работа №{\thelabworkcounter}}
	\textit{Тема: <<{#1}>>}
	%\vspace{10mm}
	\stepcounter{labworkcounter}
}

\newcommand{\labtask}{
	\stepcounter{labtaskcounter}%Чтобы не с нуля начинать
	\subsection*{Задание \thelabtaskcounter}
}

\newcommand{\reservedtasks}{
	\subsection*{Резервные задания}
}

\newcommand{\labworkquestions}{
	\subsection*{Вопросы к работе}
}

\newcommand{\codeexample}[2]{
	\lstinputlisting[label=#2,caption={#1}]{cpp-examples/#2.cpp}
}

\newcommand{\codesnippet}[1]{
	\lstinputlisting[caption={}]{#1.cpp}
}


Исходные тексты (в том числе файлы с кодом) доступны по адресам:

https://github.com/nickkolok/didact-tp/

https://gitlab.com/nickkolok/didact-tp/

Хэш-ревизия сборки (хэш последнего коммита git):
\input{.gitrevision}

Дата и время сборки:
\input{.builddate}

Распространяется на условиях свободной лицензии GNU GPLv3


\tableofcontents

\addcontentsline{toc}{chapter}{Лабораторные работы}

\labwork{Работа со строками как с массивами символов}

\labtask

Пользователем вводится строка (возможно, содержащая пробелы).
Произвести над ней заданные операции.

Указание. Словом считаеся последовательность малых или больших латинских букв A-Z, a-z.

\begin{enumerate}
	\item
		Подсчитать количество гласных букв (a,e,o,o,u,y).
	\item
		Подсчитать количество биграмм <<ab>>.
	\item
		Инвертировать регистр букв.
	\item
		Превратить большие согласные буквы в маленькие.
	\item
		Удалить из строки все html-тэги, т. е. подстроки, начинающиеся с < и заканчивающиеся ближайшей > либо концом строки.
	\item
		Иногда при наборе текста в начале слова набирающий не успевает вовремя отпустить Shift, и получается нечто вроде <<TExt>>.
		Исправить все слова, начинающиеся со сдвоенной согласной буквы.
	\item
		Иногда при наборе текста набирающий, желая поставить многоточие, забывает поставить третью точку в нём, получая нечто вроде <<Text..>>.
		Исправить такие ситуации.
	\item
		Эмоциональные школьники зачастую пишут много восклицательных знаков подряд.
		Везде, где количество подряд идущих восклицательных знаков превышает три, остальные удалить.
	\item
		Перевернуть строку.
	\item
		Заменить все вхождения подстроки 'ck' на 'kk'.
	\item
		По типографским правилам набора между запятой и предшествующим словом пробел не ставится.
		Найти все такие лишние пробелы и убрать их.
	\item
		По типографским правилам набора после запятой ставится пробел.
		Расставить недостающие пробелы.		
	\item
		Вычислить длину наибольшего фрагмента текста, заключённого между запятыми.
	\item
		Проверить, есть ли в строке закрывающая скобка <<)>>, идущая раньше, чем первая из открывающих.
	\item
		Проверить, что количество открывающих скобок <<(>> в строке соответствует количеству закрывающих <<)>>.

\end{enumerate}

\labtask

Программу, написанную в задании 1, видоизменить так, чтобы она содержала функцию, обрабатывающую строку.
Название функции должно быть адекватным.
Функция не должна самостоятельно выводить что-либо на экран.
Функция не должна изменять переданную строку.

Указание. При необходимости для копирования строк воспользуйтесь функцией strcpy.

\labtask

Покрыть функцию, написанную в задании 2, тестами,
т. е. вставить в программу код, перед началом её выполнения убеждающийся в правильности выдаваемых написанной функцией результатов
на некоторых специально подобранных характерных примерах исходных данных,
для которых требуемый результат известен.

\labtask

Измерить среднее время выполнения набора тестов, составленных в задании 3, с помощью их многократного циклического повторения.

Указание. Пример измерения времени дан в листинге \ref{timecount}


\reservedtasks

\begin{enumerate}
	\item
		Вычислить длину наибольшего фрагмента текста, заключённого между запятыми и не содержащего знаков конца предложения.
	\item
		Подсчитать, сколько слов написано ЗаБоРчИкОм
	\item
		Выяснить, является ли строка <<перевёртышем>>, например, как <<Аргентина манит негра>>.
		Регистр букв и небуквенные символы не учитывать.
\end{enumerate}


\labtask

Используя многократное (в цикле) выполнение тестов, написанных в задании 3, определить, какой прирост производительности даёт вызов функции strlen не при проверке условия завершения цикла, а до начала выполнения цикла с сохранением значения в переменную.




Лабораторная работа № 2

Тема <<Передача строк в функции и возвращение строк из функций>>

Задание 1.

Составьте программу, которая запрашивает у пользователя натуральное число, а затем выводит это число и согласованное с ним слово, соответствующее Вашему варианту.

Например, если Ваше слово <<гриб>>, а пользователь ввёл число 55, программа должна вывести сообщение <<55 грибов>>, а если пользователь ввёл число 21, то <<21 гриб>>.

Указание. Использовать деление с остатком, конструкции if-else и/или switch-case.

Слова:

\begin{enumerate}

\item стол

\item окно

\item  герань

\item  земля

\item  стул

\item  роза 

\item  звезда

\item  яблоко

\item  рулон

\item  гора

\item  башня

\item  окно

\item  абрикос

\item  картина

\item  провод



\end{enumerate}


Задание 2.

Выделить существенный код в функцию, принимающую число и необходимое количество вариантов строковой переменной и возвращающую требуемую строковую переменную.

Задание 3.

Покрыть юнит-тестами функцию, написанную Вами при выполнении задания 2, измерить время её выполнения.

Задание 4.

Используя написанную в задании 2 функцию, вывести аналогичные фразы для слов.

\begin{enumerate}

\item 	тетрадь, карандаш

\item 	стена, кирпич

\item 	дерево, лист

\item 	учитель, ученик

\item 	дом, окно

\item 	машина, колесо

\item 	мать, сын

\item 	человек, жизнь

\item 	закон, врач

\item 	песня, слово

\item 	сказка, ложь

\item 	 свет, частица

\item 	снег, лопата

\item 	война, кровь

\item 	дым, свет


\end{enumerate}

Пример работы программы:

Введите число:

5

5 грибов

5 оленей

5 лисиц  


\labwork{Работа с указателями}

\labtask

Заведите целочисленную переменную $a$, получите указатель на неё и далее обращайтесь к ней только через указатель.
Заведите динамический указатель и выделите новый участок памяти для хранения числа $d$.
Запросите у пользователя числа $a$ и $d$.
Выведите значение выражения, соответствующего Вашему варианту.

\begin{enumerate}

\item $a^3-2ad$

\item $3a^2-5d$

\item  $7a-d^3$

\item  $4ad^2+2a^3$

\item  $3ad+7d^2$

\item  $1-d^3+a^2$

\item  $4d^4+a^2$

\item  $5d-3a^4$

\item  $2a^2+ad-1$

\item  $6a^2d-2d^2$

\item  $3ad^2-7a^2$

\item  $a^4+3ad^2$

\item  $a^3d-2a$

\item  $a^2d+6ad^2$

\item  $5a+7d^3$

\end{enumerate}


\labtask

Выполните указанные вычисления.
Значения переменных, стоящих в правой части формулы, вводятся пользователем в том порядке, в котором эти переменные встречаются в формуле, слева направо, сверху вниз.
Для сохранения значений в переменную используйте прямое обращение к переменной, для получения значений переменной - обращение через указатель.


\begin{enumerate}

\item $\rho =\frac{p}{gh}$

\item $a=\frac{v^2}{r}$

\item  $s=v_0t+\frac{at^2}{2}$

\item  $p=\frac{mg}{S}$

\item  $\omega =\frac{2\pi }{T}$

\item  $n=\frac{N}{V}$

\item  $\eta =1-\frac{T_2}{T_1}$

\item  $\varphi =k\frac{Q}{r}$

\item  $C=\frac{\epsilon \epsilon_0S}{d}$

\item  $F=\frac{kQ_1}{\epsilon r^2}$

\item  $P=\frac{A}{t}$

\item  $r=\frac{mv}{qB}$

\item  $R=\frac{R_1R_2}{R_1+R_2}$

\item  $I=\frac{U}{R}$

\item  $n=\frac{n_2}{n_1}$

\end{enumerate}

К лабораторной работе предусмотрены дополнительные вопросы.


\labwork{Работа с одномерными массивами}

\labtask

С клавиатуры вводится длина целочисленного массива.
Затем пользователь выбирает, ввести массив с клавиатуры или сгенерировать случайным образом.

\begin{enumerate}

	\item Найти минимальный элемент среди всех элементов массива.

	\item Найти максимальный элемент среди всех элементов массива.

	\item Найти минимум модулей всех элементов массива.

	\item Найти максимум модулей всех элементов массива.

	\item Найти минимум модулей разностей всех соседних элементов массива.

	\item  Найти максимум модулей разностей всех соседних элементов массива.

	\item Найти разность между максимальным и минимальным элементами массива.

	\item Найти среднее арифметическое всех элементов массива.

	\item Найти среднее арифметическое модулей всех элементов массива.

	\item Найти сумму всех элементов массива.

	\item Найти произведение всех элементов массива.

	\item Найти сумму квадратов всех элементов массива.

	\item Найти сумму кубов всех элементов массива.

	\item Найти произведение всех ненулевых элементов массива.

	\item Найти количество всех ненулевых элементов массива.

\end{enumerate}


\labtask

\begin{enumerate}

	\item Найти количество единиц среди всех элементов массива.

	\item Найти количество максимальных элементов среди всех элементов массива.

	\item Найти количество минимальных элементов среди всех элементов массива.

	\item Найти количество максимальных элементов по модулю среди всех элементов массива.

	\item Найти номер первого максимального элемента среди всех элементов массива.

	\item Найти номер первого минимального элемента среди всех элементов массива.

	\item Найти номер первого максимального элемента по модулю среди всех элементов массива.

	\item Найти номер последнего максимального элемента среди всех элементов массива.

	\item Найти номер последнего минимального элемента среди всех элементов массива.

	\item Найти номер последнего максимального по модулю элемента среди всех элементов массива.

	\item Найти номер последнего минимального по модулю элемента среди всех элементов массива.

	\item Найти номер первого нулевого элемента массива.

	\item Найти номер первого ненулевого элемента массива.

	\item Найти номер первого положительного элемента массива.

	\item Найти номер первого отрицательного элемента массива.

\end{enumerate}

\reservedtasks

\begin{enumerate}

	\item Найти номер первого минимального элемента по модулю среди всех элементов массива.

	\item Найти количество минимальных элементов по модулю среди всех элементов массива.

\end{enumerate}


\labtask

Выделите существенный код в функции, напишите тесты (где это возможно).

Указание. Не требуется писать тесты для функции генерации случайного массива и функции ввода массива пользователем. 

\labtask

Доработать написанную в предыдущем задании программу так, чтобы она наряду с массивом и методом его ввода запрашивала у пользователя границы обрабатываемого подмассива (могла обрабатывать не весь массив целиком, а некоторую его часть), т. е. индексы (номера) элемента, с которого начинать обработку, и индекс элемента, на котором закончить обработку.

\labworkquestions

\begin{enumerate}

	\item
		Как передать массив в функцию?
	\item
		В каких случаях для передачи массива в функцию недостаточно передать указатель?
	\item
		Как вернуть массив из функции? Достаточно ли вернуть указатель?
	\item
		Как передать в функцию массив, полученный из данного отбрасыванием $n$ первых элементов?
\end{enumerate}




\labwork{Работа с одномерными массивами, часть 2}

\labtask

С клавиатуры вводится длина целочисленного массива.
Сгенерировать целочисленный массив указанной длины, заполнив его случайными числами в диапазоне от -20 до 20 включительно.
Сгенерированный массив вывести на экран.
После этого на основе полученного массива сформировать новый массив в соответствии с номером Вашего варианта.
Новый массив вывести на экран.

Указание. При выполнении работы разрешается использовать и модифицировать функции, написанные в лабораторной работе №4.

\begin{enumerate}

	\item
		Из элементов, отличающихся от минимального не более, чем на 2.

	\item
		Из элементов, отличающихся от максимального не менее, чем на 3.

	\item
		Из элементов, отличающихся от минимального по модулю элемента более, чем на 5.

	\item
		Из элементов, отличающихся от максимального по модулю элемента менее, чем на 4.

	\item
		Из номеров тех элементов, которые не больше, чем  минимум модулей разностей всех соседних элементов массива.

	\item
		Из номеров тех элементов, которые не меньше, чем максимум модулей разностей всех соседних элементов массива.

	\item
		Из номеров тех элементов, которые меньше, чем разность между максимальным и минимальным элементами массива.

	\item
		Из номеров тех элементов, которые больше, чем среднее арифметическое всех элементов массива.

	\item
		Из элементов, которые больше, чем среднее арифметическое модулей всех элементов массива.

	\item
		Из тех элементов, которые больше, чем последняя цифра суммы всех элементов массива.

	\item
		Из номеров тех элементов, которые меньше, чем последняя цифра произведения всех элементов массива.

	\item
		Из тех элементов, которые больше, чем последняя цифра суммы квадратов всех элементов массива.

	\item
		Из номеров тех элементов, которые меньше, чем последняя цифра суммы кубов всех элементов массива.

	\item
		Из номеров тех элементов, которые меньше, чем произведение всех ненулевых элементов массива.

	\item
		Из тех элементов, которые больше, чем количество всех ненулевых элементов массива.

\end{enumerate}


\labtask

Выделите существенный код в функции, напишите тесты (где это возможно).

Указание. Не требуется писать тесты для функции генерации случайного массива и функции ввода массива пользователем. 

Указание. При возвращении массивов из функции придерживаться соглашения: нулевой элемент массива хранит его длину.
При передаче массива в функцию придерживаться соглашения: размер массива передаётся отдельно от самого массива.
При определении индексов элементов элемент, хранящий длину, не учитывать, изменить указатель с помощью арифметики указателей.

\labworkquestions

\begin{enumerate}

	\item
		Как создать массив требуемой длины?
	\item
		Как сгенерировать случайное число в указанном диапазоне?
	\item
		Как объявить функцию, возвращающую массив значений типа \textbf{double}?
	\item
		Как инициализировать массив?
	\item
		Почему иногда бывает удобно при возвращении массивов из функции придерживаться соглашения: нулевой элемент массива хранит его длину?
		Как работать с таким <<надставленным>> массивом впоследствии?
\end{enumerate}

\typerrors
№\ref{TE_avoidable-overflow},
№\ref{TE_if-return-return}


\labwork{Рекурсивная обработка массивов. Измерение времени работы функции}

\labtask

На основе каждой из функций обработки массива, написанных Вами в лабораторной работе №4, составьте по две функции:
одну --- обрабатывающую массив в цикле, другую --- обрабатывающую массив рекурсивно.
С помощью тестов, написанных Вами в той же лабораторной работе, убедитесь в корректной работе всех четырёх функций.

Указание. Пример рекурсивной обработки массива дан в листинге \ref{minimum-even-recursive}.

\labtask

С помощью функции генерации случайного массива, написанной Вами ранее, сгенерируйте 10240 случайных массивов длиной 1024.
С помощью цикла измерьте среднее время работы каждой из функций.
Выведите на экран сумму всех возвращённых функциями значений.

Указание. Пример измерения времени дан в листинге \ref{timecount}.

Сделайте вывод о соотношении времени работы рекурсивной и циклической функций, включите его в отчёт.
Запустите программу несколько раз.
Сделайте вывод о стабильности или нестабильности этого отношения и абсолютных величин затрачиваемого времени, включите его в отчёт.
Укажите причины такой стабильности или нестабильности для абсолютного времени и для отношения.

\labtask

Предполагая, что длина массива не превосходит 120 элементов, напишите несколько функций, использующих в качестве размера массива и итератора цикла переменные различных известных Вам типов (тип размера и тип итератора должны совпадать).
Измерьте соотношение времени работы функций и абсолютное время работы функций.

Сделайте вывод о наиболее быстрой функции, включите его в отчёт.

\labtask

Выполните то же, но для массивов из 8 элементов.

\labworkquestions
\begin{enumerate}
	\item
		Что такое рекурсия?
	\item
		Чем прямая рекурсия отличается от косвенной?
	\item
		Какие виды циклов существуют в языке С++?
	\item
		Может ли размер массива быть отрицательным?
	\item
		Как измерить время работы программы?
	\item
		Какой заголовочный файл следует подключить для измерения времени работы программы или её части?
	\item
		За что отвечает константа \textbf{CLOCKS\_PER\_SEC}?
		Чему она равна на используемой связке компилятор+ОС?
\end{enumerate}

\typerrors
№\ref{TE_for-from-0-instead-of-1}%

\labwork{Перегрузка функций. Шаблоны функций. Параметры функций по умолчанию.}

\labtask

Каждую из функций обработки массива, написанных Вами в лабораторной работе №4, перегрузите для работы с типами \textbf{int}, \textbf{float} и \textbf{unsigned long int}.
Другие необходимые функции (например, генерацию массива) также перегрузите.
Ввод массива пользователем предусматривать не требуется.
Сформируйте прототипы функций, расположите написанные Вами функции после \textbf{main}.
С помощью тестов убедитесь в корректной работе всех четырёх функций.


\labtask

Функции, написанные Вами в предыдущем задании, объедините с помощью шаблонов функций.
Возможно, некоторые функции объединить не получится в силу специфики реализации.
Такие функции оставьте перегруженными.

Указание. Примеры перегрузки функций и написания шаблонов функций даны в листинге \ref{function-templates}.

С помощью тестов, написанных Вами в предыдущем задании, убедитесь в правильности работы функций.


\labworkquestions
\begin{enumerate}
	\item
		Что такое <<перегрузка функций>>?
	\item
		Чем могут отличаться друг от друга перегруженные функции?
	\item
		Что такое шаблон функции?
	\item
		Что является параметрами шаблона функции?
	\item
		Когда при вызове функции, написанной в виде шаблона, необходимо указывать параметры шаблона?
	\item
		Каким образом шаблону функции передаются его параметры?
\end{enumerate}


\labwork{Интеграция с существующей системой}

Эта лабораторная работа отличается от предыдущих.
Технические навыки программирования, которых требует её выполнение, значительно ниже, чем в предыдущих,
а значительную часть алгоритмов Вы можете просто скопировать из предыдущих заданий.
С другой стороны, в этой работе Вам потребуется написать программу, которая не останется в стенах учебной лаборатории, а найдёт реальное применение в составе образовательного OpenSource-проекта <<Час ЕГЭ>>.
Необходимость интеграции с существующей системой накладывает и определённые ограничения на набор используемых возможностей языка.

Перед началом выполнения работы изучите листинг \ref{CHAS-EGE-task}.

\labtask

Изучите выданную преподавателем задачу.
Выделите и запишите в отчёт параметры, которые можно изменять автоматически, например:
\begin{itemize}
	\item
		Числа
	\item
		Имена
	\item
		Названия предметов
\end{itemize}

Составьте таблицу, в которой укажите название параметра, тип, допустимые значения и значение, используемое в задаче.

\labtask

Составьте (в используемой Вами среде разработки на С++) программу, которая выводит:
\begin{enumerate}
	\item
		Строку <<Задание:>>
	\item
		Перевод строки
	\item
		Текст задачи без переводов строки внутри него
	\item
		Перевод строки
	\item
		Строку <<Ответ:>>
	\item
		Перевод строки
	\item
		Ответ на задание --- целое число или конечную десятичную дробь
	\item
		Перевод строки
\end{enumerate}

Помните: неверно выделенный параметр или некорректная область его изменения --- ошибка.

\labtask

Придайте необходимую степень случайности всем параметрам, выделенным в предыдущем задании.
Листинг \ref{CHAS-EGE-task} содержит примеры таких изменений для числа, строки и слова, с падежами которого нужно работать.
Вы можете использовать функции из этого листинга или написать свои.

\labtask

С помощью ключевого слова \textbf{typedef} дайте новое название типу \textbf{char}.
С помощью этого названия дайте новое название типу \textbf{char*}.
В дальнейшем избегайте использования типов \textbf{char} и \textbf{char*} по их общепринятому названию,
соответствующие части программы измените.

\labtask

Дайте новые названия  всем типам-указателям, которые возвращают используемые Вами функции.
Объявите функции с помощью этих новых названий.
Если Вы не используете функции, возвращающие указатели, в отчёте укажите этот факт.

\labtask

Перейдите по адресу https://www.math.vsu.ru/chas-ege/sh/otladka.html , вставьте составленную Вами программу в поле ввода,
замените тип \textbf{char} на тип \textbf{wchar\_t} и убедитесь, что интеграция Вашей программы с тренажёром через стандартный поток вывода прошла успешно.

\labtask

Перепишите в отчёт несколько заданий, сгенерированных Вашей программой, и их решения.
Убедитесь, что полученный Вами ответ совпадает с выдаваемым написанной Вами программой.

\labworkquestions

\begin{enumerate}
	\item
		Для чего нужно ключевое слово \textbf{typedef}?
\end{enumerate}




\labwork{Способы передачи аргументов функции}

\labtask

Для фигуры, соответствующей Вашему варианту (см. табл.), напишите void-функцию, по известным величинам вычисляющую искомые.
Эта функция не должна ничего выводить на экран.
Передачу результатов вычисления из функции организуйте через указатели.
Напишите программу, которая запрашивает у пользователя известные величины, с помощью написанной функции вычисляет искомые и выводит результаты вычислений на экран.
Гарантируется корректность входных данных, т. е. существование геометрической фигуры с заданными параметрами.


\begin{tabular}{|c|c|p{0.2\linewidth}|p{0.4\linewidth}|} \hline
Вариант & Геометрическая \linebreak фигура & Известные величины & Искомые величины \\ \hline
1  & Квадрат & Длина диагонали & Длина стороны, площадь, периметр \\ \hline
2  & Квадрат & Длина стороны & Длина диагонали, площадь, периметр \\ \hline
3  & Квадрат & Площадь & Длина диагонали, длина стороны, периметр \\ \hline
4  & Квадрат & Периметр & Длина диагонали, длина стороны, площадь \\ \hline

5  & Круг    & Длина окружности & Радиус, диаметр, площадь круга \\ \hline
6  & Круг    & Радиус & Длина окружности, диаметр, площадь круга \\ \hline
7  & Круг    & Диаметр & Длина окружности, радиус, площадь круга \\ \hline
8  & Круг    & Площадь круга & Длина окружности, радиус, диаметр \\ \hline

9  & Ромб    & Длины диагоналей & Длина стороны, площадь, периметр \\ \hline
10 & Ромб    & Длина стороны, \ \linebreak площадь & Длины диагоналей, периметр \\ \hline
11 & Ромб    & Периметр, площадь & Длины диагоналей, длина стороны \\ \hline

12 & Прямоугольник & Периметр, площадь & Длина диагонали, длины сторон \\ \hline
13 & Прямоугольник & Длины сторон & Длина диагонали, периметр, площадь \\ \hline
14 & Прямоугольник & {Длина одной из сторон, \linebreak длина диагонали} & Длина другой стороны, периметр, площадь \\ \hline
15 & Прямоугольник & Длина диагонали, \ \linebreak площадь & Длины сторон, периметр \\ \hline

\end{tabular}

\labtask

Программу, написанную в задании 1, переработайте так, чтобы передача результатов вычисления из функции осуществлялась по ссылке.

\labworkquestions

\begin{enumerate}
	\item
		Какие способы передачи аргументов функции Вы знаете?
	\item
		Приведите пример функции, принимающей аргументы по ссылке, и её вызова.
		Приведите несколко примеров некорректного вызова этой же функции и объясните, в чём заключается некорректность.
	\item
		Чем отличается передача аргумента по ссылке от передачи по указателю?
	\item
		Чем отличается передача аргумента по ссылке от передачи по значению?
\end{enumerate}




\labwork{Непрямоугольные двумерные массивы. Кэширование}

\labtask

Пользователем вводятся натуральные числа $t$ и $s$, не превосходящие 1024.
Напишите программу, которая запрашивает у пользователя эти числа.
В случае, если введённые пользователем данные корректны, программа вычисляет значение функции $f(t,s)$, выводит его на экран и снова запрашивает новую пару чисел.
Если данные некорректны, то программа выдаёт сообщение об этом и завершает свою работу.

\begin{enumerate}
	\item
		$f(t,s)=\frac{t}{s} \ln(1+(t+s)^5) \cdot \cos^2(t+s)$
	\item
		$f(t,s)=(t+s)e^{\sin(t+s) \cdot \cos^3(t-s)}$
	\item
		$f(t,s)=(t-s)^3 e^{\sin(t+s)} \cdot \cos^2(t+s)$
	\item
		$f(t,s)=e^{\sin(t-s) \cdot \cos(t-s})$
	\item
		$f(t,s)=\frac{t}{s}e^{\sin(t-s) \cdot \cos(t-s)}$
	\item
		$f(t,s)=(t+s)\tg \frac{\sin(t+s)}{1+\cos^4(t-s)}$
	\item
		$f(t,s)=(t-s)\tg \frac{\sin(t+s)}{1+\cos^4(t-s)}$
	\item
		$f(t,s)=(t-s)\tg \frac{\sin(t-s)}{1+\cos^4(t-s)}$
	\item
		$f(t,s)=(t-s)e^{\ln(1+(t+s)^2) \cdot \cos^2(t+s)}$
	\item
		$f(t,s)=e^{\sin(t+s) \cdot \cos(t-s)}$
	\item
		$f(t,s)=\left(\ln\frac{t}{s}\right)e^{\log_7 (1+(t-s)^2) \cdot \cos^2(t+s)}$
	\item
		$f(t,s)=(t-s)^4 e^{\tg(t+s) \cdot \cos^2(t+s)}$
	\item
		$f(t,s)=\log_{t+s} \left(\frac{t}{s}\right)e^{\ln(1+(t-s)^2) \cdot \cos^2(t+s)}$
	\item
		$f(t,s)=\ctg^3(t+s) \cdot \sh(t-s)$
	\item
		$f(t,s)=\ctg^3(t-s) \cdot \ch(t-s)$
\end{enumerate}

\labtask

Программу, написанную в предыдущем задании, модифицируйте так, чтобы, когда это возможно, использовались результаты предыдущих вычислений.
Возможность использования обоснуйте.
Экономьте память.

\labtask

Функции, написанные в двух предыдущих заданиях, покройте тестами.
Предусмотрите запуск тестов перед приглашением пользователю ввести данные и вывод сообщения об успешности прохождения тестов.

\labtask

Измерьте среднее время вычисления значения функции на большом количестве случайных значений с использованием результатов предыдущих вычислений и без него.
В конце выведите на экран сумму всех вычисленных значений функции.
Сделайте выводы.

\labworkquestions

\begin{enumerate}
	\item
		Что такое кэширование?
	\item
		Можно ли создать статический двумерный массив непрямоугольной формы?
	\item
		Зачем при измерении времени выполнения функции нужно выводить сумму вычисленных значений?
\end{enumerate}




\labwork{Обработка двумерных массивов}

\labtask

С клавиатуры вводятся размеры целочисленного двумерного массива.
Затем пользователь выбирает, ввести массив с клавиатуры или сгенерировать случайным образом.

Указание. Разрешается использовать и модифицировать функции, написанные при выполнении лабораторной работы №4.

\begin{enumerate}
	\item Найти минимальный элемент среди всех элементов массива.

	\item Найти максимальный элемент среди всех элементов массива.

	\item Найти минимум модулей всех элементов массива.

	\item Найти максимум модулей всех элементов массива.

	\item Найти минимум модулей разностей всех соседних элементов массива.

	\item  Найти максимум модулей разностей всех соседних элементов массива.

	\item Найти разность между максимальным и минимальным элементами массива.

	\item Найти среднее арифметическое всех элементов массива.

	\item Найти среднее арифметическое модулей всех элементов массива.

	\item Найти сумму всех элементов массива.

	\item Найти произведение всех элементов массива.

	\item Найти сумму квадратов всех элементов массива.

	\item Найти сумму кубов всех элементов массива.

	\item Найти произведение всех ненулевых элементов массива.

	\item Найти количество всех ненулевых элементов массива.

\end{enumerate}


\labtask

Выделите существенный код в функции, напишите тесты (где это возможно).

\labworkquestions

\begin{enumerate}
	\item
		Как передать двумерный массив в функцию?
	\item
		Как передать двумерный массив из функции?
	\item
		Как хранится в памяти ЭВМ двумерный массив?
	\item
		Сколько байт памяти занимает двумерный массив размера 5 на 5 целочисленных двубайтных переменных на ЭВМ, память которой адресуется 8-байтными указателями? 
\end{enumerate}




\labwork{Чтение и запись информации в двоичные и текстовые файлы}

\labtask

Модифицируйте программу, написанную Вами в задании 2 лабораторной работы №10: предусмотрите чтение (перед запуском) и запись (перед завершением работы) кэша в двоичный файл \textbf{cache.bin}.
В случае первого запуска программы, т. е. если файла кэша нет, выдать сообщение об этом и сформировать пустой кэш.

\labtask

Модифицируйте программу, написанную при выполнении задания 2 лабораторной работы №5: массив  считывается из файла с именем, которое указывает пользователь.
Длина массива указывается в этом же файле.
Если указанного пользователем файла не существует, следует выдавать сообщение об ошибке и запрашивать у пользователя другое имя файла до тех пор, пока чтение не пройдёт корректно.
После того, как файл успешно прочитан, пользователь выбирает, вывести ли информацию на экран или записать в файл;
в последнем случае пользователь также указывает имя файла.


\labworkquestions

\begin{enumerate}
	\item
		Чем текстовый файл отличается от двоичного?
	\item
		Как прочесть массив из текстового файла?
\end{enumerate}




\labwork{Возведение числа в заранее известную степень}

\labtask

Написать программу, запрашивающую у пользователя число и возводящую его в заданную (заранее известную) степень
без использования стандартной функции \textbf{pow}.

Степени:

\begin{enumerate}

\item 5

\item 6

\item  7

\item  9

\item  10

\item  11

\item  12

\item  14

\item  15

\item  17

\item  19

\item  20

\item  21

\item  22

\item  23

\end{enumerate}


\labtask

Выделить существенный код в функцию, принимающую число и возвращающую результат возведения в степень.

\labtask

Оптимизировать написанную Вами в задании 2 функцию по времени выполнения.



\newpage
\phantomsection
\addcontentsline{toc}{chapter}{Требования к лабораторным работам}
\chapter*{Требования к лабораторным работам}
Данный раздел содержить список требований, необходимых для того, чтобы Ваша лабораторная работа была успешно рассмотрена.

\begin{enumerate}
	\item
		Требования к содержанию лабораторной работы, отправляемой на проверку.
	\begin{enumerate}
		\item
			Запрещается сдавать отчёт, в который не входит ничего, кроме титульного листа и/или формулировок заданий.
		\item
			Запрещается повторно сдавать уже проверенный преподавателем отчёт.
	\end{enumerate}

\end{enumerate}

\begin{enumerate}
	\item
		Требования к оформлению лабораторной работы, отправляемой на проверку.
	\begin{enumerate}
		\item
			Отчёт по лабораторной работе отправляется в виде одного файла в формате ГОСТ Р ИСО/МЭК 26300-2010 (ODT, LibreOffice: libreoffice.org).
			Файл имеет имя вида \textbf{lab2.1\_Avdeev\_003.odt},
			где вместо \textbf{2} --- номер семестра, вместо \textbf{1} --- номер лабораторной работы в семестре,
			вместо \textbf{Avdeev} --- Ваша фамилия в именительном падеже транслитом,
			вместо \textbf{003} --- номер попытки сдачи лабораторной работы.
		\item
			Файлы с программным кодом, отправляемые на рассмотрение, имеют имя вида \textbf{lab2.1-5\_Avdeev\_003.cpp},
			где вместо \textbf{5} --- номер задания, код к которому содержится в файле.
			Файл должен иметь кодировку UTF-8.
	\end{enumerate}

\end{enumerate}




\newpage
\phantomsection
\addcontentsline{toc}{chapter}{Правила оформления программного кода}
\chapter*{Правила оформления программного кода}
Правила оформления программного кода (Code Style, соглашение о стиле кода, стилевой регламент) --- это соглашение о том, как оформлять код в рамках проекта или нескольких проектов.

Казалось бы, что необходимость думать не только о логике работы программы, но и об её оформлении в соответствии с некими правилами должна мешать программисту и повышать трудозатраты на написание кода, но на деле это не так.

Во-первых, привыкнуть к выполнению нескольких простых правил --- это несложно и быстро.

Во-вторых, единообразное оформление значительно облегчает восприятие кода, что особенно важно в тех случаях,
когда над проектом работает несколько человек.
Обычно в таких случаях говорят: <<Код проекта должен выглядеть так, как будто его писал один человек>>.

В-третьих, наличие правил позволяет программисту не задумываться над тем, как лучше оформить код в данном случае, и это экономит время и силы.
Это вообще достаточно общий принцип: чем более сильные условия наложены, тем лучше результат;
студенты математических направлений не раз наблюдали его в действии на примере многочисленных теорем.

В-четвёртых, это в некоторой мере защищает код программы от бессмысленных изменений (например, от случайно поставленного пробела в конце строки).
Отсутствие бессмысленных (и зачастую не замечаемых человеком) изменений очень важно для корректной работы систем контроля версий (сокращённо СКВ; классический пример СКВ --- git), так как они, как правило, отслеживают изменения построчно.
Code Style борется с ситуациями, когда в <<журнал>> (репозиторий) СКВ записываются изменения, которые ничего не значат.

Итак, при выполнении лабораторных работ студентам следует руководствоваться нижеследующим Code Style.
Заметим, что он значительно мягче большинства регламентов, принятых в крупных проектах,
и может в некоторых случаях дополняться преподавателем.


\begin{enumerate}
	\item
		Отступы.

		Отступы в коде необходимы, так как существенно облегчают восприятие структуры программы.
		\begin{enumerate}
			\item
				Отступы делаются клавишей TAB.
			\item
				Отступ у первой строки файла отсутствует.
			\item
				При расстановке отступов пустые строки не учитываются.
			\item
				Для изменения отступа у $n$-й строки по сравнению с предыдущей ($(n-1)$-й) должны быть веские причины.
			\item
				Если $n$-я строка заканчивается любой открывающей скобкой $($, $[$ или $\{$,
				то $(n+1)$-я строка смещается на один отступ вправо.
			\item
				Если $n$-я строка начинается любой закрывающей скобкой $)$, $]$ или $\}$,
				то она смещается на один отступ влево.

				При выполнении этих правил строка, начинающаяся с некоторой закрывающей скобки, оказывается ровно под строкой,
				которая закончилась парной открывающей, а всё, что между ними --- как бы внутри.

				Правильно:
				\codesnippet{code-style-snippets/tabs-right-1}
%				\begin{lstlisting}{Language={C++},caption={},frame=single,numbers=left}
%\shortlisting{
%cin>>n;
%if(n<0){
%	cout<<"n не должно быть отрицательным, используется абсолютная величина"<<endl;
%	n=-n;
%}
%cout<<sqrt(n);
%}
%				\end{lstlisting}
		\end{enumerate}
	\item
\end{enumerate}








\newpage
\phantomsection
\addcontentsline{toc}{chapter}{Листинги программ}
\chapter*{Листинги программ}
\codeexample{Измерение времени выполнения кода}{timecount}
\newpage
\codeexample{Передача строки в функцию и возвращение строки функцией}{string-char-to-function}
\codeexample{Замена "ck" на "k}{string-char-processing}
\codeexample{Указатели и функции}{functions-and-pointers}
\newpage
\codeexample{Оператор \textbf{sizeof}}{sizeof}
\newpage
\codeexample{Кэширование значений, возвращаемых функцией}{fibo-cache}
\codeexample{Арифметические и логические операции с указателями}{pointers-arithmetic}
\codeexample{Рекурсивная и нерекурсивная обработка массивов}{minimum-even-recursive}
\codeexample{Шаблоны функций и перегрузка функций}{function-templates}
\newpage
\codeexample{Генерация задач. Ключевое слово \textbf{typedef}}{CHAS-EGE-task}
\newpage
\codeexample{Передача параметров по ссылке и значение ссылочного параметра по умолчанию}{swap-with-global}
\codeexample{Исключения различного типа и их перехват (необязательный материал)}{exceptions-average-negative}
\codeexample{Обработка строк и поразрядные операторы}{bit-operators}
\newpage
\codeexample{Поразрядный сдвиг}{binary-translations}
\newpage
\codeexample{Обмен переменных с помощью поразрядных операторов}{bit-swap}
\newpage
\codeexample{XOR-шифрование}{xor-encoding}
\codeexample{Чтение и запись из двоичных файлов}{binary-files}
\codeexample{Отображение шестнадцатеричного представления чисел с плавающей запятой}{viewing-float-inside}
\newpage
\codeexample{Макросы и приведение типов}{double-epsilon}
\codeexample{Работа со структурами в стиле C (необязательный материал)}{structs-c-style}
\newpage
\codeexample{Ссылочные и постоянные параметры функции}{const-and-ref-parameters}
\newpage
\codeexample{Условно-универсальный кроссплатформенный шаблон программы}{universal-crossplatform-template}
\newpage
\codeexample{\textbf{if}-\textbf{if}-\textbf{else}, или о пользе фигурных скобок}{if-if-else-braces}
\newpage
\codeexample{Инициализация генератора случайных чисел}{srand}



\newpage
\chapter*{Типовые ошибки}
\addcontentsline{toc}{chapter}{Типовые ошибки}
%\begin{typerror}
	Использование знакового типа там, где достаточно беззнакового.

	Как правило, с беззнаковыми числами процессору легче работать (проще сравнивать и т.д .), кроме того, это даёт увеличение максимального допустимого значения вдвое по сравнению со знаковым аналогом.

	С другой стороны, в каждом конкретном случае следует учитывать, планируется ли изменять рассматриваемую переменную и может ли при таком изменении её значение стать отрицательным.

	Неправильно:
	\codesnippet{typical-errors-snippets/signed-1-wrong}

	Правильно:
	\codesnippet{typical-errors-snippets/signed-1-right}


	Следует уделять внимание наличию знака и при написании функций.
	Программист, использующий функцию, ожидает, что она будет корректно работать при любых значениях входных параметров, поэтому отсутствие знака следует указывать явно (и при попытке передать в такую функцию знаковую переменную компилятор выдаст предупреждение).
	Кроме того, программист, использующий функцию, должен предусмотреть корректную работу разрабатываемой программы при любом возвращаемом функцией значении.
	Если функция возвращает неотрицательное, т. е. беззнаковое значение, это следует указывать явно.

	Неправильно:
	\codesnippet{typical-errors-snippets/signed-2-wrong}
	Ни размер массива, ни количество заданных элементов в нём, ни индекс элемента не могут быть отрицательными.
	Правильно:
	\codesnippet{typical-errors-snippets/signed-2-right}
	Как правило, достаточно добавить модификатор \textbf{unsigned}.
	Для размеров массивов и индексов элементов рекомендуется использовать специальный встроенный тип \textbf{size\_t}.
	В ряде случаев более высокую производительность показывает \textbf{unsigned short}.
	С  другой стороны, функция, которая по смыслу должна возвращать положительное значение, может использовать возврат отрицательных значений для сообщения об ошибке:
	\codesnippet{typical-errors-snippets/signed-2-note1}

	
\end{typerror}

\begin{typerror}[Операторы непосредственно после \textbf{return}, \textbf{break}, \textbf{continue} или \textbf{throw}]
	\label{TE_operators-after-return-break-continue}

	Операторы, расположенные непосредственно после \textbf{return}, \textbf{break}, \textbf{continue} или \textbf{throw},
	т. е. в том же блоке, не выполнятся никогда.

	Неправильно:
	\codesnippet{typical-errors-snippets/operators-after-return-break-continue-1-wrong}

	Правильно:
	\codesnippet{typical-errors-snippets/operators-after-return-break-continue-1-right}

	Неправильно:
	\codesnippet{typical-errors-snippets/operators-after-return-break-continue-2-wrong}

	Правильно:
	\codesnippet{typical-errors-snippets/operators-after-return-break-continue-2-right}

\end  {typerror}

\begin{typerror}
	Некорректное сравнение беззнакового выражения со знаковым.

	Следует избегать операций сравнения, которых можно избежать, зная, что некоторые переменные неотрицательны.



	Неправильно:
	\codesnippet{typical-errors-snippets/signed-with-unsigned-comparison-wrong}

	Сообщение <<\textbf{Введено отрицательное число, используется модуль}>> никогда не будет выведено.

	
\end{typerror}

\begin{typerror}[Получение указателя на переменную до её объявления]
	\label{TE_pointer-before-declaration}

	С этой ошибкой программа может не компилироваться.

	Неправильно:
	\codesnippet{typical-errors-snippets/pointer-before-declaration-1-wrong}
	Правильно:
	\codesnippet{typical-errors-snippets/pointer-before-declaration-1-right}

	Вообще не следует использовать переменную выше той строки, в которой она объявлена.
\end{typerror}

\begin{typerror}[Использование оператора побитового исключающего ИЛИ (XOR) \textbf{\^} для возведения в степень]
	\label{TE_xor-as-power}

	С этой ошибкой программа компилируется, но работает неправильно.

	В языке C++ оператор \textbf{\^} предназначен не для возведения в степень,
	как можно было бы подумать, а для побитового исключающего ИЛИ.
	Кроме того, он имеет более низкий приоритет, чем умножение.

	Неправильно:
	\codesnippet{typical-errors-snippets/xor-as-power-1-wrong}

	Правильно:
	\codesnippet{typical-errors-snippets/xor-as-power-1-right}

	Для использования функции возведения в степень \textbf{pow} требуется подключение заголовочного файла \textbf{<cmath>}.
\end{typerror}

\begin{typerror}[Использование конструкции \textbf{if-return-else}]
	\label{TE_if-return-else}

	Если в управляющей конструкции \textbf{if} в блоке, выполняемом при соблюдении условия, встретился оператор \textbf{return}, то ключевое слово \textbf{else} является избыточным: оператор \textbf{return} и так передаёт управление вовне функции.

	Неправильно:
	\codesnippet{typical-errors-snippets/if-return-else-1-wrong}

	Правильно:
	\codesnippet{typical-errors-snippets/if-return-else-1-right}

	В этом правиле, которое может показаться странным на первый взгляд, есть глубокая сермяжная правда.
	В реальных проектах зачастую функция, прежде, чем приступить к выполнению основной обработки данных, проверяет некие вырожденные случаи, убеждается в корректности переданных данных и т. д.
	Написание \textbf{else}-блока (с соответствующими отступами!) достаточно сильно затруднило бы читаемость таких программ.

	Сравните:
	\codesnippet{typical-errors-snippets/if-return-else-2-right}

	и

	\codesnippet{typical-errors-snippets/if-return-else-2-wrong}

	Несмотря на то, что в учебных программах количество таких проверок обычно невелико,
	полезно привыкать к восприятию кода без избыточных \textbf{else}.

	Вспомним, что при записи математических формул тоже используется форма
	с множественными <<если>>, но почти без <<иначе>>.
	Например:
	$$
		\sgn(x) = \left\{\begin{array}{rl}
		               1, & \mbox{ если } x > 0 \\
		               0, & \mbox{ если } x = 0 \\
		              -1, & \mbox{ если } x < 0 \\
		\end{array}\right.
	$$
	а не
	$$
		\sgn(x) = \left\{\begin{array}{rl}
		               1, & \mbox{ если } x > 0, \mbox{ иначе } \\
		               0, & \mbox{ если } x = 0, \mbox{ иначе } \\
		              -1. &  \\
		\end{array}\right.
	$$
	Заметим, однако, что дословная реализация первого варианта будет тоже избыточной.
	Достаточно двух \textbf{if} и трёх \textbf{return}, т.е. примерно так:
	$$
		\sgn(x) = \left\{\begin{array}{rl}
		               1 , & \mbox{ если } x > 0 \\
		               0 , & \mbox{ если } x = 0 \\
		              -1\, & \mbox{ во всех остальных случаях. }\\
		\end{array}\right.
	$$
\end{typerror}

\begin{typerror}
	Дублирование вычислений в линейных подалгоритмах.

	Ошибки этого типа, как правило, не приводят к некорректной работе программы;
	более того, зачастую современные компиляторы нивелируют вызванное такой ошибкой падение производительности.
	Обычно устранение дублирующихся вычислений повышает как производительность программы, так и её читаемость.

	В следующем примере инкремент переменной \textbf{i} следует выполнить до обращения к элементу массива.

	Неправильно:
	\codesnippet{typical-errors-snippets/duplicate-calculations-1-wrong}

	Правильно:
	\codesnippet{typical-errors-snippets/duplicate-calculations-1-right}

	Бывают и более интересные примеры.
	Например, в следующем коде разумно использовать остаток от деления на 100 для вычисления остатка от деления на 10 вместо того, чтобы заставлять ЭВМ заново выполнять многократное циклическое вычитание:

	Неправильно:
	\codesnippet{typical-errors-snippets/duplicate-calculations-2-wrong}

	Правильно:
	\codesnippet{typical-errors-snippets/duplicate-calculations-2-right}

	В этом примере деление с остатком можно выполнить один раз:

	Неправильно:
	\codesnippet{typical-errors-snippets/duplicate-calculations-3-wrong}

	Правильно:
	\codesnippet{typical-errors-snippets/duplicate-calculations-3-right}

	
\end{typerror}

\begin{typerror}
	Нарушения соответствия операторов \textbf{new} и \textbf{delete}.

	Ошибки этого типа, как правило, не приводят к некорректной работе программы;
	более того, зачастую современные компиляторы работают корректно, тем не менее, стандарт языка предусматривает в таком случае неопределённое поведение.

	Основное правило:
	если при операторе \textbf{new} стояли квадратные скобки \textbf{[ ]}, то и при соответствующем операторе \textbf{delete} должны быть квадратные скобки \textbf{[ ]};
	если же при \textbf{new} их не было, то и при \textbf{delete} быть не должно.

	Неправильно:
	\codesnippet{typical-errors-snippets/delete-brackets-1-wrong}

	Правильно:
	\codesnippet{typical-errors-snippets/delete-brackets-1-right}

	Удаление массива:

	Неправильно:
	\codesnippet{typical-errors-snippets/delete-brackets-2-wrong}

	Правильно:
	\codesnippet{typical-errors-snippets/delete-brackets-2-right}
	
\end{typerror}

\begin{typerror}
	\label{TE_avoidable-overflow}
	Опасность переполнения, которой можно избежать.

	Рассмотрим следующий код, который, по замыслу автора, должен находить последнюю цифру произведения всех элементов массива:
	\codesnippet{typical-errors-snippets/avoidable-overflow-1-wrong}

	На массивах небольшой длины и с небольшими элементами код работает корректно.
	Однако если чисел много или они достаточно большие, то может произойти переполнение \textbf{int}.
	Избежать этого можно, переписав код вот так:
	\codesnippet{typical-errors-snippets/avoidable-overflow-1-right}
	Записать такой вариант нам позволяет несложный факт из теории чисел: остаток произведения равен произведению остатков, доказать который мы оставляем читателю самостоятельно.
	Мы теряем в производительности за счёт выполнения деления с остатком на каждой итерации, зато выигрываем в надёжности.
\end  {typerror}

\begin{typerror}[Избыточный условный оператор при \textbf{return}]
	\label{TE_if-return-return}

	Таких операторов следует избегать: они затрудняют читаемость кода,
	хотя их негативное действие на производительность и нивелируется большинством современных компиляторов.

	Неправильно:
	\codesnippet{typical-errors-snippets/if-return-return-1-wrong}

	Неправильно:
	\codesnippet{typical-errors-snippets/if-return-return-2-wrong}

	Во втором случае к тому же налицо типовая ошибка №\ref{TE_if-return-else}.

	Правильно:
	\codesnippet{typical-errors-snippets/if-return-return-1-right}

	
\end{typerror}

\begin{typerror}
	\label{TE_duplicate-calculations-for}
	Повторяющиеся вычисления в цикле.
	
	Вычисления, результат которых не изменяется на всём протяжении цикла, следует выносить за его пределы,
	подобно тому, как в математике можно вынести за знак суммы константу-множитель, не зависящую от переменной суммирования:
	$$
		\sum_{i=1}^{n}\left(f(x_i)\cdot\max\limits_{x\in[a;b]}f'(x)\right) = \max\limits_{x\in[a;b]}f'(x)\cdot\sum_{i=1}^{n}f(x_i)
	$$


	Неправильно:
	\codesnippet{typical-errors-snippets/duplicate-calculations-for-1-wrong}

	Негативное действие подобных элементарных на производительность нивелируется большинством современных компиляторов,
	тем не менее, их следует избегать, поскольку с оптимизацией более сложных случаев компилятор может и не справиться,
	если ему не очевидно, изменяется ли результат вычисляемого выражения на протяжении цикла.

	В примере выше длина строки на протяжении цикла не изменяется,
	однако вызов функции \textbf{strlen} формально производится перед каждым выполнением тела цикла,
	поскольку она содержится в условии продолжения.

	Если считать, что длина строки равна $n$,
	а функция \textbf{strlen} для определения длины строки каждый раз просматривает один символ за другим,
	пока не встретит терминальный нуль \textbf{\textbackslash0}, т.е. каждый раз просматривает $n$ символов,
	то в итоге время, затрачиваемое на выполнения этого фрагмента кода, составит $O(n^2)$.

	Правильно:
	\codesnippet{typical-errors-snippets/duplicate-calculations-for-1-right}

	Здесь время выполнения лучше --- оно составляет $O(n)$, поскольку опеределение длины строки выполняется лишь однажды.
	
	Заметим кстати, что отдельного прохода по строке с целью вычисления её длины можно и вовсе избежать, например, так:
	\codesnippet{typical-errors-snippets/duplicate-calculations-for-2-right}

	Здесь, как только будет достигнут нуль-символ в конце строки, цикл прервётся.
	Большинство современных компиляторов сводят первый фрагмент кода к третьему самостоятельно (типовой случай цикла),
	но на это нельзя полагаться.

\end{typerror}




\newpage
\chapter*{Таблица аналогий языков Паскаль и C++}
\addcontentsline{toc}{chapter}{Таблица аналогий языков Паскаль и C++}
\begin{table}[ph]

\begin{tabular}{|l|l|m{8cm}|}
\hline
			Паскаль
&
			C++
&
			Пояснение
\\\hline
			\barecodesnippet{code-snippets/analog-begin.pas}{language=Pascal}
&
			\barecodesnippet{code-snippets/analog-begin.cpp}{language=C++}
&
	\multirow{2}{8cm}{
			Открывающая и закрывающая операторные скобки соответственно.
			Позволяют группировать другие операторы в циклах и т. д.
	}
\\[0.2cm]\cline{1-2}
			\barecodesnippet{code-snippets/analog-end.pas}{language=Pascal}
&
			\barecodesnippet{code-snippets/analog-end.cpp}{language=C++}
&
\\[0.2cm]\hline
			\barecodesnippet{code-snippets/analog-var.pas}{language=Pascal}
&
			\barecodesnippet{code-snippets/analog-var.cpp}{language=C++}
&
			В C++, в отличие от Паскаля, переменные можно объявлять почти в любом месте программы,
			а не обязательно в специальном блоке перед \textbf{begin}.
\\\hline
			\barecodesnippet{code-snippets/analog-read.pas}{language=Pascal}
&
			\barecodesnippet{code-snippets/analog-read.cpp}{language=C++}
&
			Можно вводить переменные как по одной, так и <<цепочками из воронок>>.
\\\hline
			\barecodesnippet{code-snippets/analog-write.pas}{language=Pascal}
&
			\barecodesnippet{code-snippets/analog-write.cpp}{language=C++}
&
			Выводить переменные тоже можно и поодиночке, и <<цепочками из воронок>>.
\\\hline
			\barecodesnippet{code-snippets/analog-writeln.pas}{language=Pascal}
&
			\barecodesnippet{code-snippets/analog-writeln.cpp}{language=C++}
&
			За перевод строки отвечает манипулятор потока \textbf{endl}.
			После него тоже можно ставить <<воронку>>:
			\textbf{cout <~\!\!\!< a <~\!\!\!< endl <~\!\!\!< b;}
			Здесь между значениями \textbf{a} и \textbf{b} будет перевод строки.
			Вместо \textbf{endl} можно использовать \textbf{"\textbackslash{n}"}.
\\\hline
			\barecodesnippet{code-snippets/analog-assignment.pas}{language=Pascal}
&
			\barecodesnippet{code-snippets/analog-assignment.cpp}{language=C++}
&
			В Паскале присваивание производится знаком \textbf{:=}~~,
			а в C++ одинарным знаком равенства.
			Важно не путать!
\\\hline
			\barecodesnippet{code-snippets/analog-comparison.pas}{language=Pascal}
&
			\barecodesnippet{code-snippets/analog-comparison.cpp}{language=C++}
&
			В Паскале сравнение производится одинарным знаком равенства,
			а в C++ двойным.
			Важно не путать!
\\\hline
			\barecodesnippet{code-snippets/analog-comment.pas}{language=Pascal}
&
			\barecodesnippet{code-snippets/analog-comment.cpp}{language=C++}
&
			Первый вариант называется многострочным комментарием, второй --- однострочным
			(всё, что после \textbf{//} и до конца строки --- комментарий,
			то есть компьютером не читается и нужно лишь для удобства человека)
\\\hline
			\barecodesnippet{code-snippets/analog-if-else.pas}{language=Pascal}
&
			\barecodesnippet{code-snippets/analog-if-else.cpp}{language=C++}
&
			Условный оператор (ветвление).
			Обратите внимание на то, что для сравнения переменных в условии в С++ используется запись
			\textbf{a==b}, а для присваивания --- запись \textbf{c=0}.
			Условие в C++ следует заключать в скобки.
			Аналог \textbf{then} отсутствует.
\\\hline
			\barecodesnippet{code-snippets/analog-for.pas}{language=Pascal}
&
			\barecodesnippet{code-snippets/analog-for.cpp}{language=C++}
&
			Цикл со счётчиком.
			Обратите внимание на сокращённое присваивание:
			\textbf{i++} означает \textbf{i=i+1}, а
			\textbf{s+=i} означает \textbf{s=s+i}
\\\hline
			\barecodesnippet{code-snippets/analog-while.pas}{language=Pascal}
&
			\barecodesnippet{code-snippets/analog-while.cpp}{language=C++}
&
			Цикл с условием (с предусловием).
			Делает то же самое, что и предыдущий, кроме присваивания \textbf{i} начального значения.
			Условие в C++ следует заключать в скобки.
\\\hline
			\barecodesnippet{code-snippets/analog-repeat.pas}{language=Pascal}
&
			\barecodesnippet{code-snippets/analog-repeat.cpp}{language=C++}
&
			Цикл с условием (с постусловием).
			Первый раз выполняется всегда, затем --- в зависимости от конечного условия.
			Условие в C++ следует заключать в скобки.
			В Паскале цикл прервётся, если условие, стоящее после \textbf{until} (англ. <<до>>) истинно,
			в С++ --- если условие, стоящее после \textbf{while} (англ. <<пока>>) ложно.
\\\hline
\end{tabular}

\end{table}

\end{document}
