\labwork{Интеграция с существующей системой}

Эта лабораторная работа отличается от предыдущих.
Технические навыки программирования, которых требует её выполнение, значительно ниже, чем в предыдущих,
а значительную часть алгоритмов Вы можете просто скопировать из предыдущих заданий.
С другой стороны, в этой работе Вам потребуется написать программу, которая не останется в стенах учебной лаборатории, а найдёт реальное применение в составе образовательного OpenSource-проекта <<Час ЕГЭ>>.
Необходимость интеграции с существующей системой накладывает и определённые ограничения на набор используемых возможностей языка.

Перед началом выполнения работы изучите листинг \ref{CHAS-EGE-task}.

\labtask

Изучите выданную преподавателем задачу.
Выделите и запишите в отчёт параметры, которые можно изменять автоматически, например:
\begin{itemize}
	\item
		Числа
	\item
		Имена
	\item
		Названия предметов
\end{itemize}

Составьте таблицу, в которой укажите название параметра, тип, допустимые значения и значение, используемое в задаче.

\labtask

Составьте (в используемой Вами среде разработки на С++) программу, которая выводит:
\begin{enumerate}
	\item
		Строку <<Задание:>>
	\item
		Перевод строки
	\item
		Текст задачи без переводов строки внутри него
	\item
		Перевод строки
	\item
		Строку <<Ответ:>>
	\item
		Перевод строки
	\item
		Ответ на задание --- целое число или конечную десятичную дробь
	\item
		Перевод строки
\end{enumerate}

Помните: неверно выделенный параметр или некорректная область его изменения --- ошибка.

\labtask

Придайте необходимую степень случайности всем параметрам, выделенным в предыдущем задании.
Листинг \ref{CHAS-EGE-task} содержит примеры таких изменений для числа, строки и слова, с падежами которого нужно работать.
Вы можете использовать функции из этого листинга или написать свои.

\labtask

С помощью ключевого слова \textbf{typedef} дайте новое название типу \textbf{char}.
С помощью этого названия дайте новое название типу \textbf{char*}.
В дальнейшем избегайте использования типов \textbf{char} и \textbf{char*} по их общепринятому названию,
соответствующие части программы измените.

\labtask

Дайте новые названия  всем типам-указателям, которые возвращают используемые Вами функции.
Объявите функции с помощью этих новых названий.
Если Вы не используете функции, возвращающие указатели, в отчёте укажите этот факт.

\labtask

Перейдите по адресу https://www.math.vsu.ru/chas-ege/sh/otladka.html , вставьте составленную Вами программу в поле ввода,
замените тип \textbf{char} на тип \textbf{wchar\_t} и убедитесь, что интеграция Вашей программы с тренажёром через стандартный поток вывода прошла успешно.

\labtask

Перепишите в отчёт несколько заданий, сгенерированных Вашей программой, и их решения.
Убедитесь, что полученный Вами ответ совпадает с выдаваемым написанной Вами программой.

\labworkquestions

\begin{enumerate}
	\item
		Для чего нужно ключевое слово \textbf{typedef}?
\end{enumerate}



