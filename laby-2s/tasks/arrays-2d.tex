\labwork{Обработка двумерных массивов}

\labtask

С клавиатуры вводятся размеры целочисленного двумерного массива.
Затем пользователь выбирает, ввести массив с клавиатуры или сгенерировать случайным образом.

Указание. Разрешается использовать и модифицировать функции, написанные при выполнении лабораторной работы №4.

\begin{enumerate}
	\item Найти минимальный элемент среди всех элементов массива.

	\item Найти максимальный элемент среди всех элементов массива.

	\item Найти минимум модулей всех элементов массива.

	\item Найти максимум модулей всех элементов массива.

	\item Найти минимум модулей разностей всех соседних элементов массива.

	\item  Найти максимум модулей разностей всех соседних элементов массива.

	\item Найти разность между максимальным и минимальным элементами массива.

	\item Найти среднее арифметическое всех элементов массива.

	\item Найти среднее арифметическое модулей всех элементов массива.

	\item Найти сумму всех элементов массива.

	\item Найти произведение всех элементов массива.

	\item Найти сумму квадратов всех элементов массива.

	\item Найти сумму кубов всех элементов массива.

	\item Найти произведение всех ненулевых элементов массива.

	\item Найти количество всех ненулевых элементов массива.

\end{enumerate}


\labtask

Выделите существенный код в функции, напишите тесты (где это возможно).

\labworkquestions

\begin{enumerate}
	\item
		Как передать двумерный массив в функцию?
	\item
		Как передать двумерный массив из функции?
	\item
		Как хранится в памяти ЭВМ двумерный массив?
	\item
		Сколько байт памяти занимает двумерный массив размера 5 на 5 целочисленных двубайтных переменных на ЭВМ, память которой адресуется 8-байтными указателями? 
\end{enumerate}



