\labwork{Работа с одномерными массивами, часть 2}

\labtask

С клавиатуры вводится длина целочисленного массива.
Сгенерировать целочисленный массив указанной длины, заполнив его случайными числами в диапазоне от -20 до 20 включительно.
Сгенерированный массив вывести на экран.
После этого на основе полученного массива сформировать новый массив в соответствии с номером Вашего варианта.
Новый массив вывести на экран.

Указание. При выполнении работы разрешается использовать и модифицировать функции, написанные в лабораторной работе №4.

\begin{enumerate}

	\item
		Из элементов, отличающихся от минимального не более, чем на 2.

	\item
		Из элементов, отличающихся от максимального не менее, чем на 3.

	\item
		Из элементов, отличающихся от минимального по модулю элемента более, чем на 5.

	\item
		Из элементов, отличающихся от максимального по модулю элемента менее, чем на 4.

	\item
		Из номеров тех элементов, которые не больше, чем  минимум модулей разностей всех соседних элементов массива.

	\item
		Из номеров тех элементов, которые не меньше, чем максимум модулей разностей всех соседних элементов массива.

	\item
		Из номеров тех элементов, которые меньше, чем разность между максимальным и минимальным элементами массива.

	\item
		Из номеров тех элементов, которые больше, чем среднее арифметическое всех элементов массива.

	\item
		Из элементов, которые больше, чем среднее арифметическое модулей всех элементов массива.

	\item
		Из тех элементов, которые больше, чем последняя цифра суммы всех элементов массива.

	\item
		Из номеров тех элементов, которые меньше, чем последняя цифра произведения всех элементов массива.

	\item
		Из тех элементов, которые больше, чем последняя цифра суммы квадратов всех элементов массива.

	\item
		Из номеров тех элементов, которые меньше, чем последняя цифра суммы кубов всех элементов массива.

	\item
		Из номеров тех элементов, которые меньше, чем произведение всех ненулевых элементов массива.

	\item
		Из тех элементов, которые больше, чем количество всех ненулевых элементов массива.

\end{enumerate}


\labtask

Выделите существенный код в функции, напишите тесты (где это возможно).

Указание. Не требуется писать тесты для функции генерации случайного массива и функции ввода массива пользователем. 

Указание. При возвращении массивов из функции придерживаться соглашения: нулевой элемент массива хранит его длину.
При передаче массива в функцию придерживаться соглашения: размер массива передаётся отдельно от самого массива.
При определении индексов элементов элемент, хранящий длину, не учитывать, изменить указатель с помощью арифметики указателей.

\labworkquestions

\begin{enumerate}

	\item
		Как создать массив требуемой длины?
	\item
		Как сгенерировать случайное число в указанном диапазоне?
	\item
		Как объявить функцию, возвращающую массив значений типа \textbf{double}?
	\item
		Как инициализировать массив?
	\item
		Почему иногда бывает удобно при возвращении массивов из функции придерживаться соглашения: нулевой элемент массива хранит его длину?
		Как работать с таким <<надставленным>> массивом впоследствии?
\end{enumerate}

\typerrors
№\ref{TE_avoidable-overflow},
№\ref{TE_if-return-return}

