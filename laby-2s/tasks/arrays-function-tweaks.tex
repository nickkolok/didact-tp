\labwork{Перегрузка функций. Шаблоны функций. Параметры функций по умолчанию.}

\labtask

Каждую из функций обработки массива, написанных Вами в лабораторной работе №4, перегрузите для работы с типами \textbf{int}, \textbf{float} и \textbf{unsigned long int}.
Другие необходимые функции (например, генерацию массива) также перегрузите.
Ввод массива пользователем предусматривать не требуется.
Сформируйте прототипы функций, расположите написанные Вами функции после \textbf{main}.
С помощью тестов убедитесь в корректной работе всех четырёх функций.


\labtask

Функции, написанные Вами в предыдущем задании, объедините с помощью шаблонов функций.
Возможно, некоторые функции объединить не получится в силу специфики реализации.
Такие функции оставьте перегруженными.

Указание. Примеры перегрузки функций и написания шаблонов функций даны в листинге \ref{function-templates}.

С помощью тестов, написанных Вами в предыдущем задании, убедитесь в правильности работы функций.


\labworkquestions
\begin{enumerate}
	\item
		Что такое <<перегрузка функций>>?
	\item
		Чем могут отличаться друг от друга перегруженные функции?
	\item
		Что такое шаблон функции?
	\item
		Что является параметрами шаблона функции?
	\item
		Когда при вызове функции, написанной в виде шаблона, необходимо указывать параметры шаблона?
	\item
		Каким образом шаблону функции передаются его параметры?
\end{enumerate}

