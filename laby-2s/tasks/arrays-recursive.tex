\labwork{Рекурсивная обработка массивов. Измерение времени работы функции}

\labtask

На основе каждой из функций обработки массива, написанных Вами в лабораторной работе №4, составьте по две функции:
одну --- обрабатывающую массив в цикле, другую --- обрабатывающую массив рекурсивно.
С помощью тестов, написанных Вами в той же лабораторной работе, убедитесь в корректной работе всех четырёх функций.

Указание. Пример рекурсивной обработки массива дан в листинге \ref{minimum-even-recursive}.

\labtask

С помощью функции генерации случайного массива, написанной Вами ранее, сгенерируйте 10240 случайных массивов длиной 1024.
С помощью цикла измерьте среднее время работы каждой из функций.
Выведите на экран сумму всех возвращённых функциями значений.

Указание. Пример измерения времени дан в листинге \ref{timecount}.

Сделайте вывод о соотношении времени работы рекурсивной и циклической функций, включите его в отчёт.
Запустите программу несколько раз.
Сделайте вывод о стабильности или нестабильности этого отношения и абсолютных величин затрачиваемого времени, включите его в отчёт.
Укажите причины такой стабильности или нестабильности для абсолютного времени и для отношения.

\labtask

Предполагая, что длина массива не превосходит 120 элементов, напишите несколько функций, использующих в качестве размера массива и итератора цикла переменные различных известных Вам типов (тип размера и тип итератора должны совпадать).
Измерьте соотношение времени работы функций и абсолютное время работы функций.

Сделайте вывод о наиболее быстрой функции, включите его в отчёт.

\labtask

Выполните то же, но для массивов из 8 элементов.

\labworkquestions
\begin{enumerate}
	\item
		Что такое рекурсия?
	\item
		Чем прямая рекурсия отличается от косвенной?
	\item
		Какие виды циклов существуют в языке С++?
	\item
		Может ли размер массива быть отрицательным?
	\item
		Как измерить время работы программы?
	\item
		Какой заголовочный файл следует подключить для измерения времени работы программы или её части?
	\item
		За что отвечает константа \textbf{CLOCKS\_PER\_SEC}?
		Чему она равна на используемой связке компилятор+ОС?
\end{enumerate}

