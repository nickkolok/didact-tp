\labwork{Работа с одномерными массивами}

\labtask

С клавиатуры вводится длина целочисленного массива.
Затем пользователь выбирает, ввести массив с клавиатуры или сгенерировать случайным образом.

\begin{enumerate}

	\item Найти минимальный элемент среди всех элементов массива.

	\item Найти максимальный элемент среди всех элементов массива.

	\item Найти минимум модулей всех элементов массива.

	\item Найти максимум модулей всех элементов массива.

	\item Найти минимум модулей разностей всех соседних элементов массива.

	\item  Найти максимум модулей разностей всех соседних элементов массива.

	\item Найти разность между максимальным и минимальным элементами массива.

	\item Найти среднее арифметическое всех элементов массива.

	\item Найти среднее арифметическое модулей всех элементов массива.

	\item Найти сумму всех элементов массива.

	\item Найти произведение всех элементов массива.

	\item Найти сумму квадратов всех элементов массива.

	\item Найти сумму кубов всех элементов массива.

	\item Найти произведение всех ненулевых элементов массива.

	\item Найти количество всех ненулевых элементов массива.

\end{enumerate}


\labtask

\begin{enumerate}

	\item Найти количество единиц среди всех элементов массива.

	\item Найти количество максимальных элементов среди всех элементов массива.

	\item Найти количество минимальных элементов среди всех элементов массива.

	\item Найти количество максимальных элементов по модулю среди всех элементов массива.

	\item Найти номер первого максимального элемента среди всех элементов массива.

	\item Найти номер первого минимального элемента среди всех элементов массива.

	\item Найти номер первого максимального элемента по модулю среди всех элементов массива.

	\item Найти номер последнего максимального элемента среди всех элементов массива.

	\item Найти номер последнего минимального элемента среди всех элементов массива.

	\item Найти номер последнего максимального по модулю элемента среди всех элементов массива.

	\item Найти номер последнего минимального по модулю элемента среди всех элементов массива.

	\item Найти номер первого нулевого элемента массива.

	\item Найти номер первого ненулевого элемента массива.

	\item Найти номер первого положительного элемента массива.

	\item Найти номер первого отрицательного элемента массива.

\end{enumerate}

\reservedtasks

\begin{enumerate}

	\item Найти номер первого минимального элемента по модулю среди всех элементов массива.

	\item Найти количество минимальных элементов по модулю среди всех элементов массива.

\end{enumerate}


\labtask

Измените программы, написаные в заданиях 1 и 2, выделив существенный код в функции.
Следите за тем, чтобы функции были специализированы: либо работали с потоками ввода-вывод, либо производили вычисления.
Напишите тесты (где это возможно).

Указание. Не требуется писать тесты для функции генерации случайного массива и функции ввода массива пользователем. 

\labtask

Доработать написанную в предыдущем задании программу (унаследованную от задания 1) так, чтобы она наряду с массивом и методом его ввода запрашивала у пользователя границы обрабатываемого подмассива (могла обрабатывать не весь массив целиком, а некоторую его часть), т. е. индексы (номера) элемента, с которого начинать обработку, и индекс элемента, на котором закончить обработку.

\labworkquestions

\begin{enumerate}

	\item
		Как передать массив в функцию?
	\item
		В каких случаях для передачи массива в функцию недостаточно передать указатель?
	\item
		Как вернуть массив из функции? Достаточно ли вернуть указатель?
	\item
		Как передать в функцию массив, полученный из данного отбрасыванием $n$ первых элементов?
\end{enumerate}



