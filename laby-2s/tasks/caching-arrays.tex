\labwork{Непрямоугольные двумерные массивы. Кэширование}

\labtask

Пользователем вводятся натуральные числа $t$ и $s$, не превосходящие 1024.
Напишите программу, которая запрашивает у пользователя эти числа.
В случае, если введённые пользователем данные корректны, программа вычисляет значение функции $f(t,s)$, выводит его на экран и снова запрашивает новую пару чисел.
Если данные некорректны, то программа выдаёт сообщение об этом и завершает свою работу.

\begin{enumerate}
	\item
		$f(t,s)=\frac{t}{s} \ln(1+(t+s)^5) \cdot \cos^2(t+s)$
	\item
		$f(t,s)=(t+s)e^{\sin(t+s) \cdot \cos^3(t-s)}$
	\item
		$f(t,s)=(t-s)^3 e^{\sin(t+s)} \cdot \cos^2(t+s)$
	\item
		$f(t,s)=e^{\sin(t-s) \cdot \cos(t-s})$
	\item
		$f(t,s)=\frac{t}{s}e^{\sin(t-s) \cdot \cos(t-s)}$
	\item
		$f(t,s)=(t+s)\tg \frac{\sin(t+s)}{1+\cos^4(t-s)}$
	\item
		$f(t,s)=(t-s)\tg \frac{\sin(t+s)}{1+\cos^4(t-s)}$
	\item
		$f(t,s)=(t-s)\tg \frac{\sin(t-s)}{1+\cos^4(t-s)}$
	\item
		$f(t,s)=(t-s)e^{\ln(1+(t+s)^2) \cdot \cos^2(t+s)}$
	\item
		$f(t,s)=e^{\sin(t+s) \cdot \cos(t-s)}$
	\item
		$f(t,s)=\left(\ln\frac{t}{s}\right)e^{\log_7 (1+(t-s)^2) \cdot \cos^2(t+s)}$
	\item
		$f(t,s)=(t-s)^4 e^{\tg(t+s) \cdot \cos^2(t+s)}$
	\item
		$f(t,s)=\log_{t+s} \left(\frac{t}{s}\right)e^{\ln(1+(t-s)^2) \cdot \cos^2(t+s)}$
	\item
		$f(t,s)=\ctg^3(t+s) \cdot \sh(t-s)$
	\item
		$f(t,s)=\ctg^3(t-s) \cdot \ch(t-s)$
\end{enumerate}

\labtask

Программу, написанную в предыдущем задании, модифицируйте так, чтобы, когда это возможно, использовались результаты предыдущих вычислений.
Возможность использования обоснуйте.
Экономьте память.

\labtask

Функции, написанные в двух предыдущих заданиях, покройте тестами.
Предусмотрите запуск тестов перед приглашением пользователю ввести данные и вывод сообщения об успешности прохождения тестов.

\labtask

Измерьте среднее время вычисления значения функции на большом количестве случайных значений с использованием результатов предыдущих вычислений и без него.
В конце выведите на экран сумму всех вычисленных значений функции.
Сделайте выводы.

\labworkquestions

\begin{enumerate}
	\item
		Что такое кэширование?
	\item
		Можно ли создать статический двумерный массив непрямоугольной формы?
	\item
		Зачем при измерении времени выполнения функции нужно выводить сумму вычисленных значений?
\end{enumerate}



