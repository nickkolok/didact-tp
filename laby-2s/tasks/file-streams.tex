\labwork{Чтение и запись информации в двоичные и текстовые файлы}

\labtask

Модифицируйте программу, написанную Вами в задании 2 лабораторной работы №10: предусмотрите чтение (перед запуском) и запись (перед завершением работы) кэша в двоичный файл \textbf{cache.bin}.
В случае первого запуска программы, т. е. если файла кэша нет, выдать сообщение об этом и сформировать пустой кэш.

\labtask

Модифицируйте программу, написанную при выполнении задания 2 лабораторной работы №5: массив  считывается из файла с именем, которое указывает пользователь.
Длина массива указывается в этом же файле.
Если указанного пользователем файла не существует, следует выдавать сообщение об ошибке и запрашивать у пользователя другое имя файла до тех пор, пока чтение не пройдёт корректно.
После того, как файл успешно прочитан, пользователь выбирает, вывести ли информацию на экран или записать в файл;
в последнем случае пользователь также указывает имя файла.


\labworkquestions

\begin{enumerate}
	\item
		Чем текстовый файл отличается от двоичного?
	\item
		Как прочесть массив из текстового файла?
\end{enumerate}



