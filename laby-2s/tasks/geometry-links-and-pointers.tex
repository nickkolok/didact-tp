\labwork{Способы передачи аргументов функции}

\labtask

Для фигуры, соответствующей Вашему варианту (см. табл.), напишите void-функцию, по известным величинам вычисляющую искомые.
Эта функция не должна ничего выводить на экран.
Передачу результатов вычисления из функции организуйте через указатели.
Напишите программу, которая запрашивает у пользователя известные величины, с помощью написанной функции вычисляет искомые и выводит результаты вычислений на экран.
Гарантируется корректность входных данных, т. е. существование геометрической фигуры с заданными параметрами.


\begin{tabular}{|c|c|p{0.2\linewidth}|p{0.4\linewidth}|} \hline
Вариант & Геометрическая \linebreak фигура & Известные величины & Искомые величины \\ \hline
1  & Квадрат & Длина диагонали & Длина стороны, площадь, периметр \\ \hline
2  & Квадрат & Длина стороны & Длина диагонали, площадь, периметр \\ \hline
3  & Квадрат & Площадь & Длина диагонали, длина стороны, периметр \\ \hline
4  & Квадрат & Периметр & Длина диагонали, длина стороны, площадь \\ \hline

5  & Круг    & Длина окружности & Радиус, диаметр, площадь круга \\ \hline
6  & Круг    & Радиус & Длина окружности, диаметр, площадь круга \\ \hline
7  & Круг    & Диаметр & Длина окружности, радиус, площадь круга \\ \hline
8  & Круг    & Площадь круга & Длина окружности, радиус, диаметр \\ \hline

9  & Ромб    & Длины диагоналей & Длина стороны, площадь, периметр \\ \hline
10 & Ромб    & Длина стороны, \ \linebreak площадь & Длины диагоналей, периметр \\ \hline
11 & Ромб    & Периметр, площадь & Длины диагоналей, длина стороны \\ \hline

12 & Прямоугольник & Периметр, площадь & Длина диагонали, длины сторон \\ \hline
13 & Прямоугольник & Длины сторон & Длина диагонали, периметр, площадь \\ \hline
14 & Прямоугольник & {Длина одной из сторон, \linebreak длина диагонали} & Длина другой стороны, периметр, площадь \\ \hline
15 & Прямоугольник & Длина диагонали, \ \linebreak площадь & Длины сторон, периметр \\ \hline

\end{tabular}

\labtask

Программу, написанную в задании 1, переработайте так, чтобы передача результатов вычисления из функции осуществлялась по ссылке.

\labworkquestions

\begin{enumerate}
	\item
		Какие способы передачи аргументов функции Вы знаете?
	\item
		Приведите пример функции, принимающей аргументы по ссылке, и её вызова.
		Приведите несколко примеров некорректного вызова этой же функции и объясните, в чём заключается некорректность.
	\item
		Чем отличается передача аргумента по ссылке от передачи по указателю?
	\item
		Чем отличается передача аргумента по ссылке от передачи по значению?
\end{enumerate}



