\labwork{Работа с указателями}

\labtask

Заведите целочисленную переменную $a$, получите указатель на неё и далее обращайтесь к ней только через указатель.
Заведите динамический указатель и выделите новый участок памяти для хранения числа $d$.
Запросите у пользователя числа $a$ и $d$.
Выведите значение выражения, соответствующего Вашему варианту.

\begin{enumerate}

\item $a^3-2ad$

\item $3a^2-5d$

\item  $7a-d^3$

\item  $4ad^2+2a^3$

\item  $3ad+7d^2$

\item  $1-d^3+a^2$

\item  $4d^4+a^2$

\item  $5d-3a^4$

\item  $2a^2+ad-1$

\item  $6a^2d-2d^2$

\item  $3ad^2-7a^2$

\item  $a^4+3ad^2$

\item  $a^3d-2a$

\item  $a^2d+6ad^2$

\item  $5a+7d^3$

\end{enumerate}


\labtask

Выполните указанные вычисления.
Значения переменных, стоящих в правой части формулы, вводятся пользователем в том порядке, в котором эти переменные встречаются в формуле, слева направо, сверху вниз.
Для сохранения значений в переменную используйте прямое обращение к переменной, для получения значений переменной --- обращение через указатель.


\begin{enumerate}

\item $\rho =\frac{p}{gh}$

\item $a=\frac{v^2}{r}$

\item  $s=v_0t+\frac{at^2}{2}$

\item  $p=\frac{mg}{S}$

\item  $\omega =\frac{2\pi }{T}$

\item  $n=\frac{N}{V}$

\item  $\eta =1-\frac{T_2}{T_1}$

\item  $\varphi =k\frac{Q}{r}$

\item  $C=\frac{\epsilon \epsilon_0S}{d}$

\item  $F=\frac{kQ_1}{\epsilon r^2}$

\item  $P=\frac{A}{t}$

\item  $r=\frac{mv}{qB}$

\item  $R=\frac{R_1R_2}{R_1+R_2}$

\item  $I=\frac{U}{R}$

\item  $n=\frac{n_2}{n_1}$

\end{enumerate}

\labworkquestions

\begin{enumerate}

	\item Что хранится в переменной типа {\bf  «указатель»}?

	\item Дан фрагмент кода

	{\bf int*** mas=new int**[7];}

	Что происходит в этой строке?
	Какой тип имеет выражение  {\bf mas[2] }?

	\item Для чего используется ключевое слово  {\bf new }?

	\item Какой тип имеет выражение  {\bf new double*[16] }?

	\item  Когда при использовании указателя не требуется вызов оператора {\bf delete} ?

	\item  Для чего предназначен оператор  {\bf delete} ?

	\item Дан фрагмент программы

		    {\bf
		       int k=0;

		       int* pk=\&k;

		       k=5;

		       cout <\!< (*pk);
				}

	Что будет выведено на экран и почему?

\end{enumerate}

\typerrors
№\ref{TE_delete-brackets}
