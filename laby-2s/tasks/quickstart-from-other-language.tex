\labwork{Быстрый старт (предполагается знание, например, JavaScript)}

\labtask

Изучив листинги №№ \ref{viewing-float-inside}, \ref{double-epsilon} и \ref{if-if-else-braces},
напишите программу, которая выводит на экран сообщение <<Hello, world!>>.

Указание.
Попробуйте запустить каждую из программ и посмотреть,
какие строки отвечают за вывод сообщений.

Указание.
Если программа не компилируется,
ещё раз просмотрите программы образцы и постарайтесь выделить общее у них.
Перенесено ли это общее в Вашу программу?


\labtask

Изучив листинг № \ref{if-if-else-braces},
составьте программу, которая запрашивает у пользователя числа $a$ и $b$,
а затем выводит на экран решение уравнения
$$
	ax=b
$$


\labtask

Составьте программу, которая запрашивает у пользователя числа $a$, $b$ и $c$,
а затем выводит на экран решение уравнения
$$
	ax^2+bx+c = 0
$$


\labtask

Составьте программу, которая запрашивает у пользователя натуральное число,
а затем выводит это число и согласованное с ним слово <<гриб>>.

Например, если  пользователь ввёл число 55, программа должна вывести сообщение <<55 грибов>>,
а если пользователь ввёл число 21, то <<21 гриб>>.

Указание.
Использовать деление с остатком и конструкцию \textbf{switch-case}.

Указание.
Информацию о конструкции \textbf{switch-case} можно легко найти в интернете.

Указание.
Ключевое слово \textbf{case} не всегда должно иметь парный \textbf{break}.

Указание.
Подумайте о возможности использования ключевого слова \textbf{default}.
