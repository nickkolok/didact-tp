\labwork{Передача строк в функции и возвращение строк из функций}

\labtask

Составьте программу, которая запрашивает у пользователя натуральное число, а затем выводит это число и согласованное с ним слово, соответствующее Вашему варианту.

Например, если Ваше слово <<гриб>>, а пользователь ввёл число 55, программа должна вывести сообщение <<55 грибов>>, а если пользователь ввёл число 21, то <<21 гриб>>.

Указание. Использовать деление с остатком, конструкции if-else и/или switch-case.

Указание. Пример передачи строки в функцию и из функции --- в листинге \ref{string-char-to-function}.

Слова:

\begin{enumerate}

\item стол

\item окно

\item  герань

\item  земля

\item  стул

\item  роза 

\item  звезда

\item  яблоко

\item  рулон

\item  гора

\item  башня

\item  окно

\item  абрикос

\item  картина

\item  провод



\end{enumerate}


\labtask

Выделить существенный код в функцию, принимающую число и необходимое количество вариантов строковой переменной и возвращающую требуемую строковую переменную.

\labtask

Покрыть юнит-тестами функцию, написанную Вами при выполнении задания 2, измерить время её выполнения.

\labtask

Используя написанную в задании 2 функцию, вывести аналогичные фразы для слов:

\begin{enumerate}

\item 	тетрадь, карандаш

\item 	стена, кирпич

\item 	дерево, лист

\item 	учитель, ученик

\item 	дом, окно

\item 	машина, колесо

\item 	мать, сын

\item 	человек, жизнь

\item 	закон, врач

\item 	песня, слово

\item 	сказка, ложь

\item 	 свет, частица

\item 	снег, лопата

\item 	война, кровь

\item 	дым, свет


\end{enumerate}

Пример работы программы:

Введите число:

5

5 грибов

5 оленей

5 лисиц  

