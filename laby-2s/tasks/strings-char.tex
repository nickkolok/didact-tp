\labwork{Работа со строками как с массивами символов}

\labtask

Пользователем вводится строка (возможно, содержащая пробелы).
Произвести над ней заданные операции.

Указание. Словом считаеся последовательность малых или больших латинских букв A-Z, a-z.

\begin{enumerate}
	\item
		Подсчитать количество гласных букв (a,e,i,o,u,y).
	\item
		Подсчитать количество биграмм <<ab>>.
	\item
		Инвертировать регистр букв.
	\item
		Превратить большие согласные буквы в маленькие.
	\item
		Удалить из строки все html-тэги, т. е. подстроки, начинающиеся с < и заканчивающиеся ближайшей > либо концом строки.
	\item
		Иногда при наборе текста в начале слова набирающий не успевает вовремя отпустить Shift, и получается нечто вроде <<TExt>>.
		Исправить все слова, начинающиеся со сдвоенной заглавной буквы.
	\item
		Иногда при наборе текста набирающий, желая поставить многоточие, забывает поставить третью точку в нём, получая нечто вроде <<Text..>>.
		Исправить такие ситуации.
	\item
		Эмоциональные школьники зачастую пишут много восклицательных знаков подряд.
		Везде, где количество подряд идущих восклицательных знаков превышает три, остальные удалить.
	\item
		Перевернуть строку.
	\item
		Заменить все вхождения подстроки 'ck' на 'kk'.
	\item
		По типографским правилам набора между запятой и предшествующим словом пробел не ставится.
		Найти все такие лишние пробелы и убрать их.
	\item
		По типографским правилам набора после запятой ставится пробел.
		Расставить недостающие пробелы.
	\item
		Вычислить длину наибольшего фрагмента текста, заключённого между запятыми.
	\item
		Проверить, есть ли в строке закрывающая скобка <<)>>, идущая раньше, чем первая из открывающих.
	\item
		Проверить, что количество открывающих скобок <<(>> в строке соответствует количеству закрывающих <<)>>.

\end{enumerate}

\labtask

Программу, написанную в задании 1, видоизменить так, чтобы она содержала функцию, обрабатывающую строку.
Название функции должно быть адекватным.
Функция не должна самостоятельно выводить что-либо на экран.
Функция не должна изменять переданную строку.

Указание. При необходимости для копирования строк воспользуйтесь функцией strcpy.

\labtask

Написать к функции, написанной в задании 2, тесты,
т. е. вставить в программу код, перед началом её выполнения убеждающийся в правильности выдаваемых написанной функцией результатов
на некоторых специально подобранных характерных примерах исходных данных,
для которых требуемый результат известен.
Предусмотреть типичные и предельные (крайние) случаи.

Указание. Пример написания тестов дан в листинге \ref{string-char-processing}.

\reservedtasks

\begin{enumerate}
	\item
		Вычислить длину наибольшего фрагмента текста, заключённого между запятыми и не содержащего знаков конца предложения.
	\item
		Подсчитать, сколько слов написано ЗаБоРчИкОм
	\item
		Выяснить, является ли строка <<перевёртышем>>, например, как <<Аргентина манит негра>>.
		Регистр букв и небуквенные символы не учитывать.
	\item
		Определить, является ли введённый пользователем символ буквой.
		Указание: использовать сравнение.
	\item
		Пользователь вводит строку.
		Вывести её первый символ.
	\item
		Пользователь вводит строку.
		Выяснить, является ли её первый символ буквой.
	\item
		Пользователь вводит строку.
		Вывести слово, с которого она начинается.
		Например, если пользователь ввёл
		\textbf{this is a string},
		то нужно вывести
		\textbf{this},
		а если
		\textbf{12345 vyshel zaychik pogulat'},
		то нужно вывести пустую строку
	\item
		Пользователь вводит строку.
		Подсчитать количество букв 'a' в ней.
	\item
		Пользователь вводит строку.
		Подсчитать количество букв 'a' и 'e' в ней.
	\item
		Пользователь вводит строку.
		Подсчитать количество букв в первом слове.
	\item
		Пользователь вводит строку.
		Подсчитать количество букв 'a' и 'e' в первом слове.
	\item
		Пользователь вводит строку и число $n$ --- номер символа в ней.
		Вывести номер последней буквы того слова, в который попадает $n$-й символ,
		либо $n-1$, если $n$-й символ не является буквой.
		Например, если введена фраза <<\textbf{You should not divide by zero!}>>
		и номер 6, то нужно вывести 10 (номер пробела после слова <<\textbf{should}>>),
		а если введён номер 3, то 2.
\end{enumerate}

\typerrors
№\ref{TE_operators-after-return-break-continue},
№\ref{TE_duplicate-calculations},
№\ref{TE_duplicate-calculations-for}%
, №\ref{TE_memory-leak}%
, №\ref{TE_duplicate-operations-if-else}%
, №\ref{TE_too-late-condition}%
, №\ref{TE_not-using-dynarray-size}%

