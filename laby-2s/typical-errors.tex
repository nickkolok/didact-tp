\begin{typerror}
	Использование знакового типа там, где достаточно беззнакового.

	Как правило, с беззнаковыми числами процессору легче работать (проще сравнивать и т.д .), кроме того, это даёт увеличение максимального допустимого значения вдвое по сравнению со знаковым аналогом.

	С другой стороны, в каждом конкретном случае следует учитывать, планируется ли изменять рассматриваемую переменную и может ли при таком изменении её значение стать отрицательным.

	Неправильно:
	\codesnippet{typical-errors-snippets/signed-1-wrong}

	Правильно:
	\codesnippet{typical-errors-snippets/signed-1-right}


	Следует уделять внимание наличию знака и при написании функций.
	Программист, использующий функцию, ожидает, что она будет корректно работать при любых значениях входных параметров, поэтому отсутствие знака следует указывать явно (и при попытке передать в такую функцию знаковую переменную компилятор выдаст предупреждение).
	Кроме того, программист, использующий функцию, должен предусмотреть корректную работу разрабатываемой программы при любом возвращаемом функцией значении.
	Если функция возвращает неотрицательное, т. е. беззнаковое значение, это следует указывать явно.

	Неправильно:
	\codesnippet{typical-errors-snippets/signed-2-wrong}
	Ни размер массива, ни количество заданных элементов в нём, ни индекс элемента не могут быть отрицательными.
	Правильно:
	\codesnippet{typical-errors-snippets/signed-2-right}
	Как правило, достаточно добавить модификатор \textbf{unsigned}.
	Для размеов массивов и индексов элементов рекомендуется использовать специальный встроенный тип \textbf{size\_t}.
	В ряде случаев более высокую производительность показывает \textbf{unsigned short}.
	С  другой стороны, функция, которая по смыслу должна возвращать положительное значение, может использовать возврат отрицательных значений для сообщения об ошибке:
	\codesnippet{typical-errors-snippets/signed-2-note1}

	
\end{typerror}
