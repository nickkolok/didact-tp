\begin{typerror}[Ненамеренное присваивание вместо сравнения]
	\label{TE_assignment-instead-of-comparison}

	Эта ошибка очень коварна.
	Она иногда встречается даже в серьёзных проектах у опытных программистов.
	Программа с этой ошибкой компилируется успешно, но в некоторых случаях работает неправильно.

	Во фрагменте кода ниже программист просто хотел сравнить два числа, вводимых пользователем:

	Неправильно:
	\codesnippet{typical-errors-snippets/assignment-instead-of-comparison-1-wrong}

	Результат, однако, весьма странен.
	Если \textbf{b} равно нулю, то программа выводит <<\textbf{Числа не равны}>>, даже если в качестве \textbf{a} ввести ноль.
	Если же \textbf{b} --- не ноль, то программа всегда выводит, что числа равны.
	Попробуем разобраться, что же происходит.

	Отлаживаем:
	\codesnippet{typical-errors-snippets/assignment-instead-of-comparison-1-debug}

	При проверке условия \textbf{if (a = b)} программа увидела присваивание
	и честно присвоила переменной \textbf{a} значение переменной \textbf{b}.
	Результатом этой операции стало значение переменной \textbf{b}
	(в этом можно убедиться, выполнив инструкцию \textbf{cout <~\!\!\!< (a = b);} ),
	после чего это число было интерпретировано как логическое выражение.

	Число интерпретируется как ложь, если оно равно нулю, и как истина в противном случае.
	Поэтому при \textbf{b}, равном нулю, считалось, что условие неверно,
	а при всех остальных \textbf{b} --- что верно.
	Результат не зависел от значения переменной \textbf{a},
	так как в ходе присваивания оно терялось, заменяясь на значение переменной \textbf{b}.

	Чтобы исправить ошибку, достаточно добавить второй знак равенства:

	Правильно:
	\codesnippet{typical-errors-snippets/assignment-instead-of-comparison-1-right}

	Ошибка настолько коварна, что некоторые программисты рекомендуют писать код <<в стиле магистра Йоды>>:
	вместо \textbf{if (a == 1)} писать \textbf{if (1 == a)},
	поскольку, если программист пропустит второй символ равенства,
	то  выражение \textbf{if (a = 1)} скомпилируется, а выражение \textbf{if (1 = a)} вызовет ошибку компилятора и тем самым привлечёт внимание.

	Заметим в заключение, что сравнение вместо присваивания --- т. е. написание \textbf{a == b;} вместо \textbf{a = b;}
	--- тоже не вызывает ошибку компиляции (но иногда вызывает предупреждение), но приводит к ошибками в работе программы,
	хотя встречается гораздо реже.
	Тем не менее, читателю не стоит думать, что два знака равенства всегда лучше, чем один.

\end{typerror}
