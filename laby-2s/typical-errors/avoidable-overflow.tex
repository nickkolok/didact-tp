\begin{typerror}
	\label{TE_avoidable-overflow}
	Опасность переполнения, которой можно избежать.

	Рассмотрим следующий код, который, по замыслу автора, должен находить последнюю цифру произведения всех элементов массива:
	\codesnippet{typical-errors-snippets/avoidable-overflow-1-wrong}

	На массивах небольшой длины и с небольшими элементами код работает корректно.
	Однако если чисел много или они достаточно большие, то может произойти переполнение \textbf{int}.
	Избежать этого можно, переписав код вот так:
	\codesnippet{typical-errors-snippets/avoidable-overflow-1-right}
	Записать такой вариант нам позволяет несложный факт из теории чисел: остаток произведения равен произведению остатков, доказать который мы оставляем читателю самостоятельно.
	Мы теряем в производительности за счёт выполнения деления с остатком на каждой итерации, зато выигрываем в надёжности.
\end  {typerror}
