\begin{typerror}[Использование запятой при множественном вводе/выводе]
	\label{TE_cin-cout-comma}

	Программисты, начинающие изучения C++ и привыкшие к синтаксису Паскаля,
	часто используют запятую при записи инструкций ввода/вывода.

	Скажем, на Паскале ввод нескольких переменных в одной инструкции реализован вот так:
	\barecodesnippet{typical-errors-snippets/cin-cout-comma-1-pascal.pas}{language=Pascal}

	И на C++ программист пытается воспроизвести эту конструкцию:
	\codesnippet{typical-errors-snippets/cin-cout-comma-1-wrong}


	В результате вводится только \textbf{a}, но ошибки не возникает!
	Дело в том, что в языке C++ запятая, разделяющая собой не аргументы функции,
	а отдельно стоящие выражения, является бинарным оператором...
	Впрочем, не будем утомлять читателя этими подробностями.
	Скажем только, что правильно будет написать
	\codesnippet{typical-errors-snippets/cin-cout-comma-1-right}

	А в случае вывода нескольких переменных --- не
	\codesnippet{typical-errors-snippets/cin-cout-comma-2-wrong}
	а
	\codesnippet{typical-errors-snippets/cin-cout-comma-2-right}

	Обратите внимание: направление <<галочек>> определяется тем, вывод происходит или ввод:
	они всегда направлены к \textbf{cout} и от \textbf{cin}.

\end{typerror}
