\begin{typerror}[Нарушения соответствия операторов \textbf{new} и \textbf{delete}]
	\label{TE_delete-brackets}

	Ошибки этого типа, как правило, не приводят к некорректной работе программы;
	более того, зачастую современные компиляторы работают корректно, тем не менее, стандарт языка предусматривает в таком случае неопределённое поведение.

	Основное правило:
	если при операторе \textbf{new} стояли квадратные скобки \textbf{[ ]}, то и при соответствующем операторе \textbf{delete} должны быть квадратные скобки \textbf{[ ]};
	если же при \textbf{new} их не было, то и при \textbf{delete} быть не должно.

	Неправильно:
	\codesnippet{typical-errors-snippets/delete-brackets-1-wrong}

	Правильно:
	\codesnippet{typical-errors-snippets/delete-brackets-1-right}

	Удаление массива:

	Неправильно:
	\codesnippet{typical-errors-snippets/delete-brackets-2-wrong}

	Правильно:
	\codesnippet{typical-errors-snippets/delete-brackets-2-right}
	
\end{typerror}
