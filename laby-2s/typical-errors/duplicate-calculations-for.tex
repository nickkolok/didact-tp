\begin{typerror}[Повторяющиеся вычисления в цикле]
	\label{TE_duplicate-calculations-for}
	
	Вычисления, результат которых не изменяется на всём протяжении цикла, следует выносить за его пределы,
	подобно тому, как в математике можно вынести за знак суммы константу-множитель, не зависящую от переменной суммирования:
	$$
		\sum_{i=1}^{n}\left(f(x_i)\cdot\max\limits_{x\in[a;b]}f'(x)\right) = \max\limits_{x\in[a;b]}f'(x)\cdot\sum_{i=1}^{n}f(x_i)
	$$


	Неправильно:
	\codesnippet{typical-errors-snippets/duplicate-calculations-for-1-wrong}

	Негативное действие подобных элементарных на производительность нивелируется большинством современных компиляторов,
	тем не менее, их следует избегать, поскольку с оптимизацией более сложных случаев компилятор может и не справиться,
	если ему не очевидно, изменяется ли результат вычисляемого выражения на протяжении цикла.

	В примере выше длина строки на протяжении цикла не изменяется,
	однако вызов функции \textbf{strlen} формально производится перед каждым выполнением тела цикла,
	поскольку она содержится в условии продолжения.

	Если считать, что длина строки равна $n$,
	а функция \textbf{strlen} для определения длины строки каждый раз просматривает один символ за другим,
	пока не встретит терминальный нуль \textbf{\textbackslash0}, т.е. каждый раз просматривает $n$ символов,
	то в итоге время, затрачиваемое на выполнения этого фрагмента кода, составит $O(n^2)$.

	Правильно:
	\codesnippet{typical-errors-snippets/duplicate-calculations-for-1-right}

	Здесь время выполнения лучше --- оно составляет $O(n)$, поскольку опеределение длины строки выполняется лишь однажды.
	
	Заметим кстати, что отдельного прохода по строке с целью вычисления её длины можно и вовсе избежать, например, так:
	\codesnippet{typical-errors-snippets/duplicate-calculations-for-2-right}

	Здесь, как только будет достигнут нуль-символ в конце строки, цикл прервётся.
	Большинство современных компиляторов сводят первый фрагмент кода к третьему самостоятельно (типовой случай цикла),
	но на это нельзя полагаться.

\end{typerror}
