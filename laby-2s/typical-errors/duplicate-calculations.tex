\begin{typerror}
	\label{TE_duplicate-calculations}
	Дублирование вычислений в линейных подалгоритмах.

	Ошибки этого типа, как правило, не приводят к некорректной работе программы;
	более того, зачастую современные компиляторы нивелируют вызванное такой ошибкой падение производительности.
	Обычно устранение дублирующихся вычислений повышает как производительность программы, так и её читаемость.

	В следующем примере инкремент переменной \textbf{i} следует выполнить до обращения к элементу массива.

	Неправильно:
	\codesnippet{typical-errors-snippets/duplicate-calculations-1-wrong}

	Правильно:
	\codesnippet{typical-errors-snippets/duplicate-calculations-1-right}

	Бывают и более интересные примеры.
	Например, в следующем коде разумно использовать остаток от деления на 100 для вычисления остатка от деления на 10 вместо того, чтобы заставлять ЭВМ заново выполнять многократное циклическое вычитание:

	Неправильно:
	\codesnippet{typical-errors-snippets/duplicate-calculations-2-wrong}

	Правильно:
	\codesnippet{typical-errors-snippets/duplicate-calculations-2-right}

	В этом примере деление с остатком можно выполнить один раз:

	Неправильно:
	\codesnippet{typical-errors-snippets/duplicate-calculations-3-wrong}

	Правильно:
	\codesnippet{typical-errors-snippets/duplicate-calculations-3-right}

	
\end{typerror}
