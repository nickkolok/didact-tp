\begin{typerror}[Избыточные итерации циклов или проверки условий ветвления]
	\label{TE_excessive-for-if}

	С этой ошибкой программа компилируется и работает корректно,
	однако работа программы занимает неоправданно большое время.
	(Иногда разница незаметна невооружённым глазом, но на больших
	размерах массивов она становится очевидной).

	Рассмотрим следующий код, который находит сумму элементов
	побочной диагонали матрицы размера \textbf{N$\times$N}:

	Неправильно:
	\codesnippet{typical-errors-snippets/excessive-for-if-1-wrong}

	Здесь прослеживается следующая логика программиста:
	рассмотрим каждую (\textbf{i}-ю) строку матрицы (первый, внешний \textbf{for}),
	в ней --- каждый (\textbf{j}-й) элемент (второй, внутренний \textbf{for}).
	Далее проверим, лежит ли данный элемент на побочной диагонали,
	исходя из номера строки и номера элемента в ней (т.е. номера столбца),
	за это отвечает условный оператор \textbf{if}.
	Если равенство \textbf{(j == N - 1 - i)} достигнуто, то элемент прибавим к сумме.

	Для выполнения приведённого выше алгоритма требуется $O(N^2)$ операций.

	Однако можно рассуждать иначе:
	в каждой строке существует ровно один элемент, лежащий на побочной диагонали,
	и мы можем явно вычислить его номер.
	Значит, нам достаточно пройти по каждой строке (один цикл \textbf{for})
	и прибавить к сумме один элемент с индексом, зависящим от номера строки.

	\codesnippet{typical-errors-snippets/excessive-for-if-1-trans}

	Для выполнения этого алгоритма требуется $O(N)$ операций.
	Таким образом, мы экономим один цикл и один условный оператор.

	Наконец, если нам гарантированы ненулевые размеры матриц,
	то исправим ТО №\ref{TE_for-from-0-instead-of-1}.

	Правильно:
	\codesnippet{typical-errors-snippets/excessive-for-if-1-right}

	Рассмотрим ещё один пример, в котором избыточным является только условный оператор.

	Следующий код выводит на экран все эелементы массива \textbf{arr} длины \textbf{size}
	с чётными индексами:

	Неправильно:
	\codesnippet{typical-errors-snippets/excessive-for-if-2-wrong}

	Мы наперёд знаем, как найти индекс каждого следующего элемента,
	который нужно вывести на экран.
	От условного оператора можно избавиться:

	Правильно:
	\codesnippet{typical-errors-snippets/excessive-for-if-2-right}

	В заключение следует сделать несколько замечаний.

	Во-первых, ТО №\ref{TE_for-from-0-instead-of-1} в данном случае
	де-факто не возникает:
	итерирование с единицы только усложнит алгоритм, не сэкономив нам присваивания,
	как экономит в случае с суммированием, поиском минимума и т.д.

	Во-вторых, выхода за границы массива не возникнет,
	поскольку увеличение \textbf{i} производится
	до проверки условия попадания \textbf{i} в границы массива.

	В-третьих, не всегда от условного оператора или второго цикла можно легко избавиться.
	Например, нам пришлось бы его использовать,
	если бы требовалось вывести на экран не элементы с чётными индексами,
	а чётные элементы (т.е. элементы с чётными значениями независимо от места, где они стоят).
	В таком случае нам заранее не известны индексы требуемых элементов.

\end{typerror}
