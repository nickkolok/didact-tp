\begin{typerror}[Использование конструкции \textbf{if-return-else}]
	\label{TE_if-return-else}

	Если в управляющей конструкции \textbf{if} в блоке, выполняемом при соблюдении условия, встретился оператор \textbf{return}, то ключевое слово \textbf{else} является избыточным: оператор \textbf{return} и так передаёт управление вовне функции.

	Неправильно:
	\codesnippet{typical-errors-snippets/if-return-else-1-wrong}

	Правильно:
	\codesnippet{typical-errors-snippets/if-return-else-1-right}

	В этом правиле, которое может показаться странным на первый взгляд, есть глубокая сермяжная правда.
	В реальных проектах зачастую функция, прежде, чем приступить к выполнению основной обработки данных, проверяет некие вырожденные случаи, убеждается в корректности переданных данных и т. д.
	Написание \textbf{else}-блока (с соответствующими отступами!) достаточно сильно затруднило бы читаемость таких программ.

	Сравните:
	\codesnippet{typical-errors-snippets/if-return-else-2-right}

	и

	\codesnippet{typical-errors-snippets/if-return-else-2-wrong}

	Несмотря на то, что в учебных программах количество таких проверок обычно невелико,
	полезно привыкать к восприятию кода без избыточных \textbf{else}.

	Вспомним, что при записи математических формул тоже используется форма
	с множественными <<если>>, но почти без <<иначе>>.
	Например:
	$$
		\sgn(x) = \left\{\begin{array}{rl}
		               1, & \mbox{ если } x > 0 \\
		               0, & \mbox{ если } x = 0 \\
		              -1, & \mbox{ если } x < 0 \\
		\end{array}\right.
	$$
	а не
	$$
		\sgn(x) = \left\{\begin{array}{rl}
		               1, & \mbox{ если } x > 0, \mbox{ иначе } \\
		               0, & \mbox{ если } x = 0, \mbox{ иначе } \\
		              -1. &  \\
		\end{array}\right.
	$$
	Заметим, однако, что дословная реализация первого варианта будет тоже избыточной.
	Достаточно двух \textbf{if} и трёх \textbf{return}, т.е. примерно так:
	$$
		\sgn(x) = \left\{\begin{array}{rl}
		               1 , & \mbox{ если } x > 0 \\
		               0 , & \mbox{ если } x = 0 \\
		              -1\, & \mbox{ во всех остальных случаях. }\\
		\end{array}\right.
	$$
\end{typerror}
