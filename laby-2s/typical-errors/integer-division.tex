\begin{typerror}[Нежелательное целочисленное деление]
	\label{TE_integer-division}

	С этой ошибкой программа компилируется, но работает неправильно.

	В языке C++ оператор применение к целым числам оператора \textbf{/}
	приводит к выполнению целочисленного деления,
	даже если результат такого деления затем используется в операциях с числами с плавающей точкой.

	В этом примере программист хотел вычислить модуль комплексного числа:


	Неправильно:
	\codesnippet{typical-errors-snippets/integer-division-1-wrong}

	Константы \textbf{1} и \textbf{2}~--- целые числа с точки зрения компилятора!
	И делить их он будет подобающим образом~--- нацело, в результате чего получит нуль.
	Любое число (корме нуля) в нулевой степени есть единица,
	и приведённый выше код будет при вычислении модуля любого комплексного чиса выдавать \textbf{1}.

	Существует несколько способов исправить эту ошибку.
	Достаточно преобразовать в число с плавающей точкой любой из операндов (или даже оба).


	Правильно:
	\codesnippet{typical-errors-snippets/integer-division-1-right}
	Правильно:
	\codesnippet{typical-errors-snippets/integer-division-2-right}
	Правильно:
	\codesnippet{typical-errors-snippets/integer-division-3-right}
	Правильно:
	\codesnippet{typical-errors-snippets/integer-division-4-right}

	Заметим, что подобная ошибка встречается и при делении двух целочисленных переменных
	(или константы и переменной)
	где-нибудь в составе большого выражения,
	результат которого предполагается числом с плавающей точкой.

\end{typerror}
