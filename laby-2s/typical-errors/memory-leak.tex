\begin{typerror}[Утечка памяти]
	\label{TE_memory-leak}

	Характерна для С++ как для языка с ручным управлением памятью.
	Не приводит к ошибкам компиляции и в учебных заданиях, как правило, никак себя не проявляет,
	так как основная проблема, порождаемая этой ошибкой --- увеличение памяти, потребляемой программой.
	Это увеличение в больших программах может быть бесконтрольным и приводить к падению или самой программы, или всей ОС.

	Неправильно:
	\codesnippet{typical-errors-snippets/memory-leak-1-wrong}

	Правильно:
	\codesnippet{typical-errors-snippets/memory-leak-1-right}

	<<Чисто динамические>> переменные тоже надо удалять:

	Неправильно:
	\codesnippet{typical-errors-snippets/memory-leak-2-wrong}

	Правильно:
	\codesnippet{typical-errors-snippets/memory-leak-2-right}

	Говоря об освобождении памяти, следует предостеречь читателя от типовой ошибки №\ref{TE_delete-brackets}.

\end{typerror}
