\begin{typerror}[Неиспользование верхней оценки длины динамического массива]
	\label{TE_not-using-dynarray-size}

	Рассмотрим функцию, принимающую строку и удаляющую из этой строки все точки.
	Возвращать функция будет результат удаления, если переданная строка непуста, и \textbf{NULL} в противном случае.

	Неправильно:
	\codesnippet{typical-errors-snippets/not-using-dynarray-size-1-wrong}

	Буфер, отводимый под функцию, имеет размер 1024 байта,
	в то время как мы можем смело утверждать, что результат удаления точек из строки никак не длиннее самой строки.
	Заменим \textbf{1024} на динамически определяемое число, равное длине строки плюс один
	(не забываем про терминальный ноль).

	Правильно:
	\codesnippet{typical-errors-snippets/not-using-dynarray-size-1-right}

\end{typerror}
