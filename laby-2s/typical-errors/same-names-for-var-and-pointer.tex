\begin{typerror}[Использование одного имени для переменной и указателя на неё]
	\label{TE_same-names-for-var-and-pointer}

	Зачастую в задачах, включаемых в лабораторные работы, студенту нужно завести переменную,
	получить на неё указатель и далее обращаться к переменной по указателю.
	Стремясь не запутаться при именовании переменных и указателей на них,
	студент пишет примерно такой код:

	Неправильно:
	\codesnippet{typical-errors-snippets/same-names-for-var-and-pointer-1-wrong}

	Или даже так:
	
	Неправильно:
	\codesnippet{typical-errors-snippets/same-names-for-var-and-pointer-2-wrong}

	Компилятор выдаёт ошибку, и вполне обоснованно:
	с его точки зрения программист требует создания двух переменных с одним именем, но разными типами:
	целое и указатель на целое.
	
	Исправить это ошибку достаточно просто: нужно изменить имя указателя.
	Авторы берут на себя смелость порекомендовать при именовании указателей придерживаться венгерской нотации,
	т.е. добавлять \textbf{p\_} к имени переменной:

	Неправильно:
	\codesnippet{typical-errors-snippets/same-names-for-var-and-pointer-1-right}
	
	Если дальнейшая программа уже написана, то для исправления ошибки достаточно всюду,
	где под \textbf{a} подразумевался указатель, заменить \textbf{a} на \textbf{p\_a}.
	
\end{typerror}
