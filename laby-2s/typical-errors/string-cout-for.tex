\begin{typerror}[Вывод строки с помощью цикла]
	\label{TE_string-cout-for}

	Как мы знаем, строка в C++ --- это частный случай массива, массив символов.
	Но строка --- это настолько часто используемая абстракция,
	что для работы с ней давно придуманы некоторые упрощения.

	В частности, обычный массив вводить нужно с помощью цикла;
	для строки придумана удобная функция \textbf{gets}.

	Оказывается, что верно и обратное:
	в отличие от массива, вывести строку можно тоже без цикла.
	Вместо того, чтобы писать:
	\codesnippet{typical-errors-snippets/string-cout-for-1-wrong}
	(не забываем про типовую ошибку №\ref{TE_duplicate-calculations-for}),
	можно написать одну строчку:
	\codesnippet{typical-errors-snippets/string-cout-for-1-right}

	Несмотря на то, что программа с выводом строки через цикл работает правильно,
	рекомендуется всё же в целях экономии времени и усилий на чтение кода
	использовать вывод с помощью одной строки.

	Подчеркнём, что во-первых,
	вывод строки циклом может быть полезен в случае,
	когда нужно вывести не всю строку, а её часть;
	во-вторых, другие массивы, как правило,
	приходится выводить с помощью цикла.
\end{typerror}
