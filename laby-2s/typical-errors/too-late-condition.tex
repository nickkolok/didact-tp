\begin{typerror}
	\label{TE_too-late-condition}
	Несвоевременно поздняя проверка корректности входных данных.

	Выход из функции по причине некорректности входных данных обычно должен производиться настолько рано,
	насколько это возможно.
	Рассмотрим функцию, принимающую строку и удаляющую из этой строки все точки.
	Возвращать функция будет результат удаления, если переданная строка непуста, и \textbf{NULL} в противном случае.

	Неправильно:
	\codesnippet{typical-errors-snippets/too-late-condition-1-wrong}

	Как видим, переменные \textbf{i} и \textbf{z} объявлены до проверки, хотя используются только после неё.
	Более того, память под новую строку \textbf{newStr} выделяется до проверки, но не освобождается вообще,
	что приводит к утечке памяти (типовая ошибка №\ref{TE_memory-leak}).

	Чтобы избежать ненужных действий, нужно проводить проверку как можно раньше.

	Правильно:
	\codesnippet{typical-errors-snippets/too-late-condition-1-right}

\end{typerror}
