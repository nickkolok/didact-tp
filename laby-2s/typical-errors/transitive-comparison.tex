\begin{typerror}[Транзитивное сравнение (сравнение трёх и более чисел одной конструкцией)]
	\label{TE_transitive-comparison}

	С этой ошибкой программа компилируется, но работает неправильно.

	Начинающий программист на C++ зачастую,
	когда нужно проверить попадание переменной в промежуток
	или равенство трёх переменных,
	копирует из математической литературы (например, учебника алгебры или конспекта по матанализу)
	такую конструкцию:

	Неправильно:
	\codesnippet{typical-errors-snippets/transitive-comparison-1-wrong}

	Компилятор, очевидно, в школе не учился и угадывать такое не умеет.
	Он сначала вычисляет значение первого сравнения
	(в случае неравенств это \textbf{a<b}).
	Если неравенство выполнено, результат понимается как \textbf{1}, иначе как \textbf{0}.
	Результат неравенства уже сравнивается с \textbf{c}.

	В качестве демонстрации эффекта можете попробовать запустить:
	\textbf{cout<\,\!<(1 == 2 == 3 == 0)<\,\!<endl;} и увидеть \textbf{1},
	т.е. выражение в скобках истинно.

	В таких случаях следует разбить выражение на бинарные сравнения, т.е. сравнения двух чисел.

	Правильно:
	\codesnippet{typical-errors-snippets/transitive-comparison-1-right}

\end{typerror}
