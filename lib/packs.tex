\documentclass[a4paper,14pt]{report} %размер бумаги устанавливаем А4, шрифт 12пунктов
\usepackage[T2A]{fontenc}
\usepackage[utf8]{inputenc}
\usepackage[english,russian]{babel} %используем русский и английский языки с переносами
\usepackage{amssymb,amsfonts,amsmath,mathtext,cite,enumerate,float,amsthm} %подключаем нужные пакеты расширений
\usepackage[unicode,colorlinks=true,linkcolor=blue]{hyperref}
\usepackage{indentfirst} % включить отступ у первого абзаца
\usepackage[dvips]{graphicx} %хотим вставлять рисунки?
\usepackage{ifthen}
\usepackage{multirow}
\usepackage{tabularx}
\graphicspath{{illustr/}}%путь к рисункам

\usepackage{pdfpages}

\makeatletter
\renewcommand{\@biblabel}[1]{#1.} % Заменяем библиографию с квадратных скобок на точку:
\makeatother %Смысл этих трёх строчек мне непонятен, но поверим "Запискам дебианщика"

\usepackage{geometry} % Меняем поля страницы.
\geometry{left=1cm}% левое поле
\geometry{right=1cm}% правое поле
\geometry{top=1cm}% верхнее поле
\geometry{bottom=2cm}% нижнее поле

\renewcommand{\theenumi}{\arabic{enumi}}% Меняем везде перечисления на цифра.цифра
\renewcommand{\labelenumi}{\arabic{enumi}}% Меняем везде перечисления на цифра.цифра
\renewcommand{\theenumii}{.\arabic{enumii}}% Меняем везде перечисления на цифра.цифра
\renewcommand{\labelenumii}{\arabic{enumi}.\arabic{enumii}.}% Меняем везде перечисления на цифра.цифра
\renewcommand{\theenumiii}{.\arabic{enumiii}}% Меняем везде перечисления на цифра.цифра
\renewcommand{\labelenumiii}{\arabic{enumi}.\arabic{enumii}.\arabic{enumiii}.}% Меняем везде перечисления на цифра.цифра

% Пакет для отображения исходного кода - с http://www.inp.nsk.su/~baldin/LaTeX/lurs-code.pdf
\usepackage{listings}
%\usepackage{listingsutf8}
% подгружаемые языки — подробнее в документации listings
\lstloadlanguages{C++,Pascal}
% Конфигурируем
\lstset{
	language=C++, % выбираем язык по умолчанию
	frame=single, % рамка
	commentstyle=\itshape\textcolor[rgb]{0.5,0.5,0.5}, % шрифт для комментариев
	stringstyle=\bfseries, % шрифт для строк
	numbers=left,              % где поставить нумерацию строк (слева\справа)
	numberstyle=\tiny,         % размер шрифта для номеров строк
	tabsize=2,                 % размер табуляции по умолчанию равен 2 пробелам
}

% А эта тёмная магия позволяет нормально работать с кириллицей в листингах
% Copyright Nikolay Avdeev aka NickKolok aka Николай Авдеев 2016

% Всем привет из снежного Воронежа! 

% This file is part of LISTINGCYR.

%    LISTINGCYR is free software: you can redistribute it and/or modify
%    it under the terms of the GNU General Public License as published by
%    the Free Software Foundation, either version 3 of the License, or
%    (at your option) any later version.

%    LISTINGCYR is distributed in the hope that it will be useful,
%    but WITHOUT ANY WARRANTY; without even the implied warranty of
%    MERCHANTABILITY or FITNESS FOR A PARTICULAR PURPOSE.  See the
%    GNU General Public License for more details.

%    You should have received a copy of the GNU General Public License
%    along with CHAS-CORRECT.  If not, see <http://www.gnu.org/licenses/>.

%  (Этот файл — часть LISTINGCYR.

%   LISTINGCYR - свободная программа: вы можете перераспространять её и/или
%   изменять её на условиях Стандартной общественной лицензии GNU в том виде,
%   в каком она была опубликована Фондом свободного программного обеспечения;
%   либо версии 3 лицензии, либо (по вашему выбору) любой более поздней
%   версии.

%   CHAS-CORRECT распространяется в надежде, что она будет полезной,
%   но БЕЗО ВСЯКИХ ГАРАНТИЙ; даже без неявной гарантии ТОВАРНОГО ВИДА
%   или ПРИГОДНОСТИ ДЛЯ ОПРЕДЕЛЕННЫХ ЦЕЛЕЙ. Подробнее см. в Стандартной
%   общественной лицензии GNU.

%   Вы должны были получить копию Стандартной общественной лицензии GNU
%   вместе с этой программой. Если это не так, см.
%   <http://www.gnu.org/licenses/>.)
%





% Юзер, помни!
% Сей файл под GNU GPLv3.
% Слинковался - открой сорцы!
% Copyright Nikolay Avdeev 2016
% nickkolok@mail.ru or avdeev@math.vsu.ru

% Пользуясь случаем, передаю привет из Воронежа товарищу @virens

%Спасибо юзеру waverider за http://dxdy.ru/topic18924-15.html
\lstset{literate=
	{А}{{\CYRA}}1
	{Б}{{\CYRB}}1
	{В}{{\CYRV}}1
	{Г}{{\CYRG}}1
	{Д}{{\CYRD}}1
	{Е}{{\CYRE}}1
	{Ё}{{\CYRYO}}1
	{Ж}{{\CYRZH}}1
	{З}{{\CYRZ}}1
	{И}{{\CYRI}}1
	{Й}{{\CYRISHRT}}1
	{К}{{\CYRK}}1
	{Л}{{\CYRL}}1
	{М}{{\CYRM}}1
	{Н}{{\CYRN}}1
	{О}{{\CYRO}}1
	{П}{{\CYRP}}1
	{Р}{{\CYRR}}1
	{С}{{\CYRS}}1
	{Т}{{\CYRT}}1
	{У}{{\CYRU}}1
	{Ф}{{\CYRF}}1
	{Х}{{\CYRH}}1
	{Ц}{{\CYRC}}1
	{Ч}{{\CYRCH}}1
	{Ш}{{\CYRSH}}1
	{Щ}{{\CYRSHCH}}1
	{Ъ}{{\CYRHRDSN}}1
	{Ы}{{\CYRERY}}1
	{Ь}{{\CYRSFTSN}}1
	{Э}{{\CYREREV}}1
	{Ю}{{\CYRYU}}1
	{Я}{{\CYRYA}}1
	{а}{{\cyra}}1
	{б}{{\cyrb}}1
	{в}{{\cyrv}}1
	{г}{{\cyrg}}1
	{д}{{\cyrd}}1
	{е}{{\cyre}}1
	{ё}{{\cyryo}}1
	{ж}{{\cyrzh}}1
	{з}{{\cyrz}}1
	{и}{{\cyri}}1
	{й}{{\cyrishrt}}1
	{к}{{\cyrk}}1
	{л}{{\cyrl}}1
	{м}{{\cyrm}}1
	{н}{{\cyrn}}1
	{о}{{\cyro}}1
	{п}{{\cyrp}}1
	{р}{{\cyrr}}1
	{с}{{\cyrs}}1
	{т}{{\cyrt}}1
	{у}{{\cyru}}1
	{ф}{{\cyrf}}1
	{х}{{\cyrh}}1
	{ц}{{\cyrc}}1
	{ч}{{\cyrch}}1
	{ш}{{\cyrsh}}1
	{щ}{{\cyrshch}}1
	{ъ}{{\cyrhrdsn}}1
	{ы}{{\cyrery}}1
	{ь}{{\cyrsftsn}}1
	{э}{{\cyrerev}}1
	{ю}{{\cyryu}}1
	{я}{{\cyrya}}1
}

% Люди, любите друг друга, используйте Linux и поступайте на матфак ВГУ!






